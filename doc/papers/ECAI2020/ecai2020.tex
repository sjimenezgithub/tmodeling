\documentclass{ecai}

\usepackage{times}
\usepackage{graphicx}
\usepackage{latexsym}

\usepackage{amssymb}
\usepackage{array}
\usepackage{multirow}

\newcommand{\tup}[1]{{\langle #1 \rangle}}
\newcommand{\pre}{\mathsf{pre}}    % precondition
\newcommand{\eff}{\mathsf{eff}}    % effect
\newcommand{\cond}{\mathsf{cond}}  % condition
\newcommand{\dur}{\mathsf{dur}}    % duration
\newcommand{\iscond}{\mathsf{is\_cond}}    % is_cond
\newcommand{\iseff}{\mathsf{is\_eff}}    % is_eff
\newcommand{\obs}{\mathsf{obs}}    % observation
\newcommand{\start}{\mathsf{start}}% start
\newcommand{\en}{\mathsf{end}}     % end
\newcommand{\til}{\mathsf{til}}    % TIL
\newcommand{\supp}{\mathsf{sup}}   % sup
\newcommand{\tim}{\mathsf{time}}   % time
\newcommand{\reqs}{\mathsf{req\_{start}}} % req_start
\newcommand{\reqe}{\mathsf{req\_{end}}}   % req_end
\newcommand{\ini}{\mathsf{init}}   % init
\newcommand{\goal}{\mathsf{goal}}  % goal


%Tamaño 7+1 pages

\begin{document}
\title{Learning Temporal Action Models via Constraint Programming}
\author{Antonio Garrido \and Sergio Jim\'enez\institute{VRAIN, Valencian Research Institute for Artificial Intelligence. Universitat Politecnica de Valencia, Spain, email: \{agarridot,serjice\}@dsic.upv.es} }
 
%\author{Name1 Surname1 \and Name2 Surname2 \and Name3 Surname3\institute{University, Country, email: somename@university.edu} }
 
\maketitle

\begin{abstract}
  We present a Constraint Programming (CP) formulation for learning action models in temporal planning scenarios beyond PDDL2.1.
  Inspired by the CP approach for temporal planning, our formulation bases on a temporal plan trace and represents observations (as time-stamped states), actions, causal-link relationships, condition threats and effect interferences. This formulation is very expressive and supports a wide range of input knowledge. It also evidences the connection between the tasks of: i) action model learning, ii) plan validation, and iii) plan synthesis. Our formulation is solver-independent, so off-the-shelf CSP solvers can be used for its resolution. Our experiments evaluate, both syntactically and semantically, the quality of the learned models under different learning scenarios and in different planning domains. 
\end{abstract}


\section{INTRODUCTION}

{\em Temporal planning} is an expressive planning model that relaxes the assumption of instantaneous actions of {\em classical planning}~\cite{geffner2013concise}. Actions in temporal planning are {\em durative}, as they have a duration, and their conditions/effects may hold/happen at different times~\cite{fox2003pddl2}. This means that {\em durative actions} can 
%be executed in parallel and 
overlap in different ways~\cite{cushing2007temporal}. Therefore, valid solutions for temporal planning instances must specify the precise time-stamp when durative actions start and end~\cite{howey2004val}.

Despite the potential of state-of-the-art planners, their application to real world problems is still somewhat limited mainly because of the difficulty of specifying correct and complete planning models~\cite{kambhampati2007model}. The more expressive the planning model, the more evident becomes this {\em knowledge acquisition bottleneck}, which jeopardizes the usability of planning technology. There are, however, growing efforts in the planning community for the machine learning of action models from sequential plans: since pioneering learning systems like ARMS~\cite{yang2007learning}, we have seen systems able to learn action models with {\em quantifiers}~\cite{AmirC08,ZhuoYHL10}, from {\em noisy} actions or states~\cite{MouraoZPS12,zhuo2013action}, from {\em null state information}~\cite{cresswell2013}, or from {\em incomplete} domain models~\cite{ZhuoK17,ZhuoNK13}. But they do not address the complex temporal planning aspects.

Most of the previous approaches are purely inductive and require large input datasets, e.g. hundreds of plan samples or observations, to compute statistically significant models. These approaches could potentially fall in learning a little from each sample, but not a complete valid model for a particular sample.
This paper follows a different approach and addresses the learning setting where one (or more) model is learned from a single sample, i.e. {\em one-shot} learning. Obviously, if hundreds of samples are available we will learn many models and return the most learned model, that is, the most repeated one.

%With the aim of understanding better the connection between the learning of durative action models, temporal planning and the validation of temporal plans, this paper follows a radically different approach and studies the singular learning scenario where just the observation of a single plan execution ({\em one-shot}) is available. Our main contribution is a solver-independent CP formulation that integrates the {\em learning} of temporal planning action models with the {\em synthesis} and the {\em validation} of temporal plans. 


As far as we know, this paper proposes the first approach for learning temporal action models. On the one hand, while learning action models for classical planning means computing the actions conditions and effects that are consistent with the input observations, learning temporal action models requires additionally: i) identifying the time-stamps (temporal annotations) of conditions and effects and, when necessary, ii) estimate the actions duration. 
This contributes with an appealing way to learn from plan traces with overlapping actions in multi-agent environments~\cite{furelos2018carpool}. 
On the other hand, our approach bases on the CP formulation of~\cite{garrido2009constraint}, which is used for planning and/or scheduling a whole plan. We keep the philosophy of using CP but, contrarily to~\cite{garrido2009constraint}, we address the inverse task now: learn the temporal action model given a plan trace. Intuitively, we are now interested in playing the designer's role, not the planner's.
Therefore, our approach also contributes with a solver-independent CP formulation that integrates the {\em learning} of temporal planning action models with their {\em synthesis} and {\em validation}.



\section{BACKGROUND}

%This section formalizes the {\em temporal planning} and the {\em constraint satisfaction} problems that we use in this work.

\subsection{Temporal planning}
\label{sec:temporalplanning}

We assume that {\em states} are factored into a set $F$ of Boolean variables. A state $s$ is a time-stamped assignment of values to all the variables in $F$. A {\em temporal planning problem} is a tuple $P=\tup{F,I,G,A}$ where the {\em initial state} $I$ is a fully observed state (i.e. a total assignment of the state variables $|I|=|F|$) time-stamped with $t=0$; $G \subseteq F$ is a conjunction of {\em goal conditions} over the variables in $F$ that defines the set of goal states; and $A$ represents the set of durative actions.
A durative action has a duration and conditions/effects on $F$ at different times~\cite{garrido2009constraint,vidal2006branching}. To compactly represent temporal planning problems, we assume that the state variables in $F$ are instantiations of a given set of predicates $\Psi$ (like in the PDDL language~\cite{younes2004ppddl1}) and that durative actions in $A$ are fully grounded from {\em operators}.

PDDL2.1 is the language for the temporal track of the International Planning Competition (IPC)~\cite{fox2003pddl2,ghallab2004automated}. A PDDL2.1 durative action $a\in A$ is defined with the following elements:

\begin{enumerate}
\item $\dur(a)$, a positive value indicating the duration of the action.

\item $\cond_s(a), \cond_i(a), \cond_e(a)$ representing the three types of action conditions. Unlike the \emph{pre}conditions of classical actions, action conditions in PDDL2.1 must hold: before $a$ is executed ({\em at start}), over the duration of $a$ (\textit{invariant/over all}) or when $a$ finishes ({\em at end}), respectively. 

\item $\eff_s(a)$ and $\eff_e(a)$ represent the two types of action effects. In PDDL2.1, effects can happen {\em at start} or {\em at end} of action $a$ respectively, and can be either positive or negative. %(i.e. asserting or retracting variables). 

\end{enumerate}


%PDDL2.1 is a restricted temporal planning model that defines the semantics of a durative action $a$ as two discrete events, $\start(a)$ and $\en(a)=\start(a)+\dur(a)$. This means that if $a$ starts on state $s$ with time-stamp $\start(a)$, then $\cond_s(a)$ must hold in $s$. Ending action $a$ in state $s'$, with time-stamp $\en(a)$, means $\cond_e(a)$ must hold in $s'$. {\em Over all} conditions must (invariantly) hold at any state between $s$ and $s'$ or, in other words, over the interval $\start(a)..\en(a)$. Likewise, {\em at start} and {\em at end} effects are instantaneously applied at states $s$ and $s'$, respectively (continuous effects are not considered in this work). 
Figure~\ref{fig:exampleactions2} shows an example of two PDDL2.1 durative actions from the {\em driverlog} domain of IPC.  \texttt{board-truck} defines a fixed duration of two time units whereas the duration of \texttt{drive-truck} depends on the driving time associated to the two given locations.

\begin{figure} 
\begin{scriptsize}  
\begin{verbatim}
(:durative-action board-truck
 :parameters (?d - driver ?t - truck ?l - location)
 :duration (= ?duration 2)
 :condition (and (at start (at ?d ?l)) 
                 (at start (empty ?t))
                 (over all (at ?t ?l)))
 :effect (and (at start (not (at ?d ?l))) 
              (at start (not (empty ?t)))
              (at end (driving ?d ?t))))


(:durative-action drive-truck
 :parameters (?t - truck ?l1 - location 
              ?l2 - location ?d - driver)
 :duration (= ?duration (driving-time ?l1 ?l2))
 :condition (and (at start (at ?t ?l1)) 
                 (at start (link ?l1 ?l2))
                 (over all (driving ?d ?t)))
 :effect (and (at start (not (at ?t ?l1))) 
              (at end (at ?t ?l2))))
\end{verbatim}
\end{scriptsize}    
%\end{tabular}
\caption{Two PDDL2.1 durative actions of the IPC-{\em driverlog} domain.}
\label{fig:exampleactions2}
\end{figure}


PDDL2.2 is an extension of PDDL2.1 that includes the notion of {\em Timed Initial Literal}~\cite{hoffmann2005}, denoted as $\til(p,t)$, and representing that variable $p\in F$ becomes true at a certain time $t>0$, independently of the actions in the plan~\cite{Edelkamp04}. TILs are useful to model {\em exogenous events}; for instance, in a logistics scenario, the 8h-20h time window when a warehouse is open can be modeled with these two timed initial literals: $\til(openWarehouse,8)$ and $\til(\neg openWarehouse,20)$.

%A {\em temporal plan} is a set of pairs $\pi=\{(a_1,t_1),(a_2,t_2)\ldots (a_n,t_n)\}$. Each pair $(a_i,t_i)$ contains a durative action $a_i$ and the time-stamp $t_i=\start(a_i)$. The execution of a temporal plan starting from a given initial state $I$ induces a state sequence formed by the union of all states $\{s_{t_i}, s_{t_i+\dur(a_i)}\}$, where there exists an initial state $s_{0}=I$, and a state $s_{end}$ that is the last state induced by the execution of the plan. Note then that sequential plans can be expressed as temporal plans but not the opposite. A {\em solution} to a given temporal planning problem $P$ is a temporal plan $\pi$ such that its execution, starting from the corresponding initial state, eventually reaches a state that meets the goal conditions, $G\subseteq s_{end}$. A solution is {\em optimal} iff it minimizes the plan {\em makespan} (i.e., the maximum $\en(a)=\start(a)+\dur(a)$ of an action in the plan).


A {\em temporal plan} is a set $\pi=\{(a_1,t_{a_1}),(a_2,t_{a_2})\ldots (a_n,t_{a_n})\}$, where each pair $(a_i,t_i)$ contains a durative action $a_i$ and the start time $t_i=\start(a_i)$. The execution of $\pi$, starting from a given initial state $I$, induces a state sequence formed by the union of all states $\{s_{t_i}, s_{t_i+\dur(a_i)}\}$, where there exists an initial state $s_{0}=I$, and a state $s_{end}$ that is the last state induced by the execution of $\pi$. 
%Classical plans can be expressed as temporal plans but not the opposite. 
A {\em solution} to $P$ is a plan $\pi$ such that its execution, starting from $s_{0}$, eventually satisfies $G\subseteq s_{end}$. 
%A solution is {\em optimal} iff it minimizes the plan {\em makespan} (i.e., the maximum $\en(a)$ of any action in the plan).
Note that classical plans can be expressed as temporal plans but not the opposite. 


\subsection{Constraint Satisfaction Problems}

A {\em Constraint Satisfaction Problem} (CSP) is a tuple $\tup{X,D,C}$, where $X$ is a set of finite-domain {\em variables}, $D$ represents the {\em domain} for each of these variables and $C$ is a set of {\em constraints} among the variables in $X$ that bound their possible values in $D$.

A {\em solution} to a CSP is an assignment of values to all variables in $X$ that is {\em consistent} with $C$.
%all constraints in $C$. 
If we do not define a metric over $X$, many solutions, i.e. different variable assignments that are consistent with the input constraints, are possible and equally valid.
%Given a CSP there may be many different solutions to that problem, i.e. different variable assignments that are consistent with the input constraints.

%A {\em cost-function} can be defined over the variables in $X$ to specify user preferences about the space of possible solutions. Given a CSP and a cost-function, then an {\em optimal solution} is a full variable assignment that is consistent with the constraints of the CSP such that it also minimizes the value of the defined cost-function.



\section{LEARNING A TEMPORAL ACTION MODEL}
\label{section:learningTemporalModels}

We formalize the task for {\em learning a temporal action model} as a tuple $\mathcal{L}=\tup{F,I,G,A?,O,C}$ where:

\begin{itemize}
	
\item $\tup{F,I,G,A?}$ is a temporal planning problem such that actions in $A?$ are {\em incomplete}. By incomplete we mean that the exact conditions+effects, their temporal annotation (\textit{at start}, \textit{over all} or \textit{at end}), and the duration of actions are unknown.
The operator, name and parameters of actions are known.  
With this regard, a predicate $p\in F$ that appears in the set of FOL interpretations over the action parameters of $a$ is considered a member of the alphabet of $a$ ($\alpha(a)$). For instance in {\em driverlog},
$\alpha(${\footnotesize \texttt{board-truck(driver1,truck1,loc1)}$)=\{$\texttt{(at driver1 loc1), (at truck1 loc1), (empty truck1), (driving driver1 truck1), (path loc1 loc1), (link loc1 loc1)}}$\}$.
Formally, we define $candidates(a)$ as the tuple $\langle\{p_i\}, \{p_i \cup \neg p_i\} \rangle$, where $p_i \in \alpha(a)$. The first set $\{p_i\}$ denotes all candidates that can be conditions of $a$, whereas the second set $\{p_i \cup \neg p_i\}$ denotes all candidates that can be effects of $a$. Without loss of generality, we learn positive conditions and positive+negative effects. 
%Note that $candidates(a)$ contains all the potential predicates that action $a$ could use; its size depends on the predicates in $F$ and the parameters of $a$, e.g. $|candidates($\texttt{board-truck(?d,?t,?l)}$)|=6*3=18$.
Note that $candidates(a)$ contains all the potential predicates that action $a$ (or the corresponding operator) could learn; its size depends on the size of $\alpha(a)$,
e.g. $|candidates($\texttt{board-truck(?d,?t,?l)}$)|=6*3=18$.


\item $O$ is the set of {\em observations} over a plan trace. Al least, this set contains a full observation of $I$ (time-stamped with $t=0$) and a final state observation, which equals $G$ (time-stamped with $t_{end}$, the {\em makespan} of the observed plan). Although $I$ represents a full state observation, the final observation can represent a full or partial state: in plan synthesis it is the partial goal state, in plan validation it is the full state to be satisfied, and in learning it is the partial or full state to be explained by the learned model.
$O$ also contains the observations over the start and/or end times of actions and, optionally, other 
time-stamped observations of traversed intermediate partial states\footnote{This work supports partial observations under the hypothesis that not all variables must be observed at any time. Observations are noiseless, which means that observed values are actual values with no uncertainty.}. Figure~\ref{fig:exampleObservations} shows an example of $O$ from the {\em driverlog} domain.


\item $C$ is an optional set of {\em constraints} that captures domain-specific expert knowledge. In this work these constraints are:

\begin{itemize}
\item Mutually-exclusive ({\em mutex}) constraints that allow us to: i) automatically deduce new observations, and ii) prune action models inconsistent with these constraints. For instance, we can provide input knowledge to avoid drivers to be in two different locations at the same time. Hence, if we learn the driver is in one location we can automatically deduce an observation (s)he is no longer in the other locations.
Figure~\ref{fig:example-statecs} shows an example of six mutex constraints for the {\em driverlog} domain. 


%\item Constraints over $candidates(a)$ to represent partially specified action models~\cite{ZhuoNK13}. For instance, we may know in advance that action {\tt board-truck} requires the {\tt driver} and the {\tt truck} to be at the same location, and the \texttt{path} and \texttt{link} are unnecessary. These constraints reduce the size of $candidates$, thus improving the learning.
\item Constraints over $candidates(a)$ to represent partially specified action models~\cite{ZhuoNK13}. For instance, we may know in advance that \texttt{path} and \texttt{link} are unnecessary for {\tt board-truck}, while \texttt{path} is unnecessary for {\tt drive-truck}.
These constraints reduce the size of $candidates(a)$, thus improving the learning.
\end{itemize}
\end{itemize}



\begin{figure}%[hbt!]
	\begin{scriptsize}    
		\begin{verbatim}
(:objects driver1 driver2 - driver
          truck1 truck2 - truck
          package1 package2 - obj
          s0 s1 s2 p1-0 p1-2 - location)
		
(:init (at driver1 s2) (at driver2 s2) (at truck1 s2)
       (empty truck1) (at truck2 s0) (empty truck2) 
       (at package1 s0) (at package2 s0)
       (path s1 p1-0) (path p1-0 s1) (path s0 p1-0) 
       (path p1-0 s0) (path s1 p1-2) (path p1-2 s1)
       (path s2 p1-2) (path p1-2 s2)
       (link s0 s1) (link s1 s0) (link s0 s2) 
       (link s2 s0) (link s2 s1) (link s1 s2))
		
(:goal (and (at driver1 s1) (at truck1 s1)))
		
(:observation :time 1 
              :start (board-truck driver1 truck1 s2)
(:observation :time 11 
              :start (drive-truck truck1 s2 s1 driver1)
...
(:observation :time 56 (at driver1 s1) (at truck1 s2))
(:observation :time 78 (at package1 s0) (at package2 s0))
	
	\end{verbatim}
	\end{scriptsize}    
	\caption{Example of $O$ containing a full state $I$, a partial state $G$, the start time of two actions, and two optional time-stamped partial states.}
	% from the {\em driverlog} domain.}
	\label{fig:exampleObservations}
\end{figure}

\begin{figure}
\begin{scriptsize}
   
\begin{tabular}{p{2.8cm}l}

$\forall$ \texttt{drv,loc1,loc2:} & $\neg$\texttt{at(drv,loc1) $\vee$ $\neg$at(drv,loc2),} \\
& $\neq$\texttt{(loc1,loc2)}\\

$\forall$ \texttt{trck,loc1,loc2:} & $\neg$\texttt{at(trck,loc1) $\vee$ $\neg$at(trck,loc2),} \\
& $\neq$\texttt{(loc1,loc2)}\\

$\forall$ \texttt{trck,drv:} & $\neg$\texttt{empty(trck) $\vee$ $\neg$driving(drv,trck)}\\

$\forall$ \texttt{drv,trck1,trck2:}  & $\neg$\texttt{driving(drv,trck1)} $\vee$\\
& $\neg$\texttt{driving(drv,trck2),} $\neq$\texttt{(trck1,trck2)}\\

$\forall$ \texttt{drvr1,drvr2,trck:}  & $\neg$\texttt{driving(drvr1,trck)} $\vee$\\
& $\neg$\texttt{driving(drvr2,trck),} $\neq$\texttt{(drvr1,drvr2)}\\

$\forall$ \texttt{drvr,loc,trck:} & $\neg$\texttt{at(drvr,loc)} $\vee$ $\neg$\texttt{driving(drvr,trck)}

\end{tabular}  
\end{scriptsize}    
\caption{Examples of six mutex constraints in $C$ for the {\em driverlog} domain with drivers (\textit{drv}), locations (\textit{loc}) and trucks (\textit{trck}).}
\label{fig:example-statecs}
\end{figure}

A {\em solution} to a learning task $\mathcal{L}$ is a fully specified model of durative actions $\mathcal{A}$ such that the conditions+effects, their temporal annotations and the duration of any action in $\mathcal{A}$ are: i) completely specified, and ii) {\em consistent} with $\mathcal{L}=\tup{F,I,G,A?,O,C}$. By consistent we mean that there exists a valid plan that exclusively contains actions in $\mathcal{A}$ and whose execution, starting in $I$, produces all the observations in $O$ at the associated time-stamps, while it satisfies all constraints in $C$, and reaches a final state that satisfies $G$.




\section{FORMULATING THE LEARNING TASK AS A CSP}
\label{section:learningAsCSP}


Given a learning task $\mathcal{L}$ as defined in Section~\ref{section:learningTemporalModels}, we automatically create a solver-independent CSP whose solution induces an action model that solves $\mathcal{L}$.
%This CSP is solver-independent and integrates previous work on temporal planning as satisfiability~\cite{garrido2009constraint,hu2007temporally,rintanen2015discretization,vidal2006branching}. 


\subsection{The variables}
For each action $a\in A?$ and predicate (condition or effect) $p\in candidates(a)$, we create the variables of Table~\ref{table:variables}. 
X1 represents the time-stamp when $a$ start, X2 when $a$ ends and X3 its duration. The values of X1, X2 and X3 can either be observed in $O$ or derived from the expression $\en(a)=\start(a)+\dur(a)$. We  model time in $\mathbb{Z}^+$ and bound all maximum times to the plan {\em makespan} ($t_{end}$ if observed in $O$). If $t_{end}$ is not observed, we consider a large enough domain for time. 
Boolean variables X4/X5 model whether $p$ is actually a condition/effect of $a$. X6.1 and X6.2 define the interval throughout condition $p$ must hold for the application of action $a$ (provided $\iscond(p,a)$=\textit{true}). X7 models a {\em causal link}, representing that action $b$ supports $p$, which is required by $a$. If $p$ is not a condition of $a$ ($\iscond(p,a)$=\textit{false}) then $\supp(p,a)$=$\emptyset$, representing an empty supporter. X8 models the time-stamp when effect $p$ happens in $a$ (provided $\iseff(p,a)$=\textit{true}).

\begin{table}
\begin{center}
\caption{The CSP variables, their domain and description.}
\begin{scriptsize}
%\begin{tabular}{llll}
\begin{tabular}{p{0.3cm}p{1.3cm}p{1.5cm}p{3.8cm}}
	\hline	
	{\bf ID} & {\bf Variable} & {\bf Domain} & {\bf Description} \\
	\hline
	X1 &$\start(a)$ & $[0..t_{end}]$ & {\em Start time} of action $a$ \\
	X2 &$\en(a)$ & $[0..t_{end}]$ & {\em End time} of action $a$ \\
	X3 &$\dur(a)$ & $[0..t_{end}]$ & {\em Duration} of action $a$ \\
	
	X4 &$\iscond(p,a)$ & $\{true,false\}$ & true if $p$ is a {\em condition} of $a$; false otherwise \\
	X5 &$\iseff(p,a)$ & $\{true,false\}$ & true if $p$ is an {\em effect} of $a$; false otherwise \\
	
	X6.1 &$\reqs(p,a)$, &  & \\ 
	X6.2 &$\reqe(p,a)$  & $[0..t_{end}]$ & Interval when action $a$ requires $p$ \\
	
	X7 &$\supp(p,a)$ & $\{b\}_{b\in A?} \cup \emptyset $&  Supporters for causal link $\tup{b,p,a}$ \\ 
	X8 &$\tim(p,a)$ & $[0..t_{end}]$ & Time when the effect $p$ of $a$ happens\\
	
\end{tabular}
\end{scriptsize}
\label{table:variables}
\end{center}
\end{table}


Additionally, we create two dummy actions:

\begin{itemize}
	
	\item $\ini$, which represents the initial state $I$ ($\start(\ini)=0$ and $\dur(\ini)=0$). It has no conditions so it has no associated variables $\iscond, \reqs, \reqe$ and $\supp$. It has as many $\iseff(p_i,\ini)$=\textit{true} and $\tim(p_i,\ini)=0$ as $p_i$ in $I$.
	
	\item $\goal$, which represents the goals $G$ ($\start(\goal)=t_{end}$ and $\dur(\goal)=0$). It has no effects so it has no $\iseff$ and $\tim$ variables. It has as many $\iscond(p_i,a)$=\textit{true}, $\supp(p_i,\goal)\neq \emptyset$ and $\reqs(p_i,\goal)=\reqe(p_i,\goal)=t_{end}$ as $p_i$ in $G$. 
\end{itemize}  


This formulation is powerful enough to model TILs and observations. A $\til(p,t)$ is analogous to $\ini$, and it is
%the initial state of a planning task.
%(both represent information that is given at a particular time, but externally to the execution of the plan). 
modeled as a dummy action that starts at time $t$ and has instantaneous duration ($\start(\til(p,t))=t$ and $\dur(\til(p,t))=0$) with no conditions and the single effect $p$ that happens at time $t$ ($\iseff(p,\til(p,t))$=\textit{true} and $\tim(p,\til(p,t))=t$). 
An observation $\obs(p,t)$ is analogous to $\goal$, and it 
% of a planning task, as they both represent conditions that must be satisfied by the execution of the plan. 
%An observation 
is modeled as a dummy action that starts at time $t$ and has instantaneous duration ($\start(\obs(p,t))=t$ and $\dur(\obs(p,t))=0$) but with only one condition $p$, which is the value observed for $p$ ($\iscond(p,\obs(p,t))$=\textit{true}, $\supp(p,\obs(p,t))\neq \emptyset$ and $\reqs(p,\obs(p,t))=\reqe(p,\obs(p,t))=t$), and no effects at all. Observations can also refer to $\start(a)$, $\en(a)$ or $\dur(a)$.




\subsection{The constraints}
\label{section:CSPconstraints}

Table~\ref{table:constraints} shows the constraints among the variables of Table~\ref{table:variables}. C1 and C2
model the end of any action, which must happen no later than $\goal$. 
C3 models that only action conditions require valid supporters.
C4 forces to have a well-defined $[\reqs,\reqe]$ interval, throughout condition $p$ is required in $a$.
C5 models that the time when $b$ supports $p$ must be before $a$ requires it because of the causal link $\tup{b,p,a}$\footnote{$\tim(p,b) < \reqs(p,a)$ and not $\leq$ because our temporal planning model assumes $\epsilon > 0$ ($\epsilon$ denotes a small tolerance that implies no collision between the time when effect $p$ is supported and when it is required, like in PDDL2.1~\cite{fox2003pddl2}). When time is modeled in $\mathbb{Z}^+$, $\epsilon=1$ so $\leq$ becomes $<$.}. 
Given a causal link $\tup{b,p,a}$, C6 avoids the {\em threat} of action $c$ deleting $p$ (threats are solved via {\em promotion} or {\em demotion}~\cite{ghallab2004automated}). 
C7 prevents action $a$ from being a supporter of $p$ when $\iseff(p,a)$=\textit{false}. 
C8 models the fact that when the same action requires and deletes $p$ the effect cannot happen before the condition. Note the $\geq$ inequality here: if one condition and one effect of the same action happen at the same time, the underlying semantics in planning considers the condition is checked instantly before the effect~\cite{fox2003pddl2}. 
C9 prevents two actions from having contradictory effects. 
C10 only applies to non-dummy actions and forces them to have at least one condition and one effect (as usual, \textit{true} is counted as 1 and \textit{false} as 0).


\begin{table*}
	\begin{center}
		\caption{The CSP constraints and their description.}	
		\begin{scriptsize}
			\begin{tabular}{p{0.1cm}p{10.2cm}p{6.4cm}}
				\hline
				{\bf ID}&{\bf Constraint}&{\bf Description}\\\hline
				C1& $\en(a)=\start(a)+\dur(a)$ & Relationship among start, end and duration of $a$ \\
				
				C2& $\en(a) \leq \start(\goal)$ & Always $\goal$ is the last action of the plan \\
				
				C3& \textbf{iff} ($\iscond(p,a)$=\textit{false}) \textbf{then} $\supp(p,a) = \emptyset$ & $p$ is not a condition of $a \iff $ the supporter of $p$ in $a$ is $\emptyset$ \\
				
				C4& \textbf{if} ($\iscond(p,a)$=\textit{true}) \textbf{then} $\reqs(p,a) \leq \reqe(p,a)$ & [$\reqs(p,a)..\reqe(p,a)$] is a valid interval\\
				
				C5& \textbf{if} ($\iseff(p,b)$=\textit{true}) \textbf{AND} ($\iscond(p,a)$=\textit{true}) \textbf{AND} ($\supp(p,a)=b$))  & Modeling the causal link $\tup{b,p,a}$: supporting $p$ before it is \\
				&\hspace{0.2cm}\textbf{then} $\tim(p,b) < \reqs(p,a)$ & required (obviously $b \neq \emptyset$) \\
				
				C6& \textbf{if} ($\iseff(p,b)$=\textit{true}) \textbf{AND} ($\iscond(p,a)$=\textit{true}) \textbf{AND} ($\iseff(\neg p,c)$=\textit{true}) \textbf{AND} ($\supp(p,a)=b$) & Solving threat of $c$ to causal link $\tup{b,p,a}$ by promotion or \\
				&\hspace{0.2cm}\textbf{AND} ($c \neq a$) \textbf{then} ($\tim(\neg p,c) < \tim(p,b)$) \textbf{OR} ($\tim(\neg p,c) > \reqe(p,a)$) & demotion (obviously $b \neq \emptyset$) \\
				
				C7& \textbf{if} ($\iseff(p,a)$=\textit{false}) \textbf{then forall} $b$ that requires $p$: $\supp(p,b) \neq a$ & $a$ cannot be a supporter of $p$ for any action $b$\\
				
				C8& \textbf{if} ($\iscond(p,a)$=\textit{true}) \textbf{AND} ($\iseff(\neg p,a)$=\textit{true}) \textbf{then} $\tim(\neg p,a) \geq \reqe(p,a)$ & $a$ requires and deletes $p$: the condition holds before the effect \\ 
				
				C9& \textbf{if} ($\iseff(p,b)$=\textit{true}) \textbf{AND} ($\iseff(\neg p,c)$=\textit{true}) \textbf{then} $\tim(p,b) \neq \tim(\neg p,c)$ & Solving effect interference at the same time ($p$ and $\neg p$) \\
				
				C10& \textbf{forall} condition $p_i$ and effect $p_j$ of $a$: $\sum \iscond(p_i,a) \geq 1$ \textbf{AND} $\sum \iseff(p_j,a) \geq 1$ & Every non-dummy action has at least one condition and effect \\
			\end{tabular}
		\end{scriptsize}	
		\label{table:constraints}
	\end{center}	
\end{table*}


%Some conditions of Table~\ref{table:constraints} are redundant. For instance C5 and C6, $\supp(p,a)=b$ means obligatorily $\iseff(p,b)=$ \textit{true}. We include them here to define an homogeneous formulation but they are not included in our implementation. For simplicity, the value of some unnecessary variables is not bounded in the table. For instance, if $\iscond(p,a)$=\textit{false}, variables $\reqs(p,a)$ and $\reqe(p,a)$ become useless.


\subsubsection*{Specific constraints for PDDL2.1}
%\label{sec:PDDL21constraints}

Our formulation is more expressive than
%accommodates a level of expressiveness beyond 
PDDL2.1. For instance, it allows conditions/effects to be at any time:
%, even outside the execution of the action; 
constraint $\reqs(p,a)=\start(a)-2$ and $\reqe(p,a)=\start(a)+2$ easily allows condition $p$ to hold in $\start(a)\pm2$.
%Likewise an effect $p\in\alpha(a)$ might also happen after the action ends e.g., 
%, and, likewise, effect $p$ happening after $a$ ends $\tim(p,a)=\en(a)+1$.

Making our formulation PDDL2.1-compliant is straightforward by adding the constraints of Table~\ref{table:21constraints} for all non-dummy actions. C11 limits the conditions to be only \emph{at start}, \emph{over all} or \emph{at end}. C12 limits the effects to happen \emph{at start} or \emph{at end}. In PDDL2.1, all actions $\{a_j\}$ grounded from the same operator share the same structure of conditions/effects. C13 guarantees this for the conditions and C14 for the effects.
C15 makes the duration of all occurrences of the same action equal (this is optional in PDDL2.1). 
C16 forces all actions to have at least one of its \textit{n}-effects \textit{at end}. Actions with only \textit{at start} effects turn the value of the duration irrelevant and they could exceed the plan makespan. Although this last constraint is not specific of PDDL2.1, it produces more rationale models for their durative actions.

\begin{table}
	\begin{center}   
		\caption{Constraints to learn PDDL2.1-compliant action models.}	
		\begin{scriptsize}
			%\begin{tabular}{ll}
			\begin{tabular}{p{0.4cm}p{7cm}}
				\hline	
				{\bf ID} &{\bf Constraint} \\ %& {\bf Description} \\
				\hline
				C11.1& ($\reqs(p,a) = \start(a)$) \textbf{OR} ($\reqs(p,a) = \en(a)$) \\% & Conditions at start\\
				C11.2& ($\reqe(p,a) = \start(a)$) \textbf{OR} ($\reqe(p,a) = \en(a)$) \\% & Conditions at end\\
				C12& ($\tim(p,a) = \start(a)$) \textbf{OR} ($\tim(p,a) = \en(a)$) \\ %& Effects at start or at end\\
				C13.1& $\forall p_i: (\forall a_j: \reqs(p_i,a_j) = \start(a_j))$ \textbf{OR} \\%& Conditions of the operator instantiations\\
				&\hspace{0.65cm}$(\forall a_j: \reqs(p_i,a_j) = \en(a_j))$ \\
				C13.2& $\forall p_i: (\forall a_j: \reqe(p_i,a_j) = \start(a_j))$ \textbf{OR} \\
				&\hspace{0.65cm}$(\forall a_j: \reqe(p_i,a_j) = \en(a_j))$ \\
				C14& $\forall p_i: (\forall a_j: \tim(p_i,a_j) = \start(a_j))$ \textbf{OR} \\%  & Effects of the operator instantiations\\
				&\hspace{0.65cm}$(\forall a_j: \tim(p_i,a_j) = \en(a_j))$ \\
				C15& $\forall a_i,a_j$ occurrences of the same action: $\dur(a_i) = \dur(a_j)$ \\ %& Duration of the operator instantiations\\
				C16 &$\sum_{i=1}^{n} \tim(p_i,a) > n \times \start(a)$ 
				
			\end{tabular}
		\end{scriptsize}
		\label{table:21constraints}
	\end{center}
\end{table}


\subsubsection*{Mutex constraints}

As seen in Section~\ref{section:learningTemporalModels}, mutex constraints in $C$ can be exploited as input knowledge to automatically deduce new observations in $\mathcal{L}$.
If two predicates $\tup{p_i, p_j}$ are mutex, they cannot hold simultaneously. This means that if $p_i$ holds, we can infer $\neg p_j$ (despite $\neg p_j$ was not actually observed). 
%Likewise, if we observe $p_j$, we can infer $\neg p_i$. 
This reasoning is specially relevant for correctly learning {\em negative effects} when there is a lack of input observations.
After all, what is the necessity to learn negative effects if they are not required nor directly observed?
Mutex reasoning helps us to fill this void by automatically inferring the observation of negated variables, which forces later to satisfy the {\em causal links} of negative variables, and improves the learned models (as we will see in Section~\ref{sec:evaluation}).
Note however that, given a $\tup{p_i, p_j}$-mutex in a durative actions setting, $\neg p_i$ does not necessarily implies $p_j$. 
See, for example, the effects \texttt{(not (at ?t ?l1))} and \texttt{(at ?t ?l2)} of action \texttt{drive-truck} in Figure~\ref{fig:exampleactions2}, which respectively happen at different times (\textit{at start} \textit{vs.} \textit{at end}).
As defined in Figure~\ref{fig:example-statecs}, these two predicates are mutex as a truck cannot be in two locations simultaneously; although 
it is valid for a truck to be, for some time (from start to end), at no location. Note this situation does not happen in classical planning, where actions have instantaneous effects and if $\tup{p_i, p_j}$ are mutex, then $p_i$ implies $\neg p_j$ and vice versa.

{\em Dynamic observations} are necessary to exploit mutex constraints at any intermediate state, even if such state was not observed at all. Reasoning on a mutex $\tup{p_i, p_j}$ means that, immediately after $a$ asserts $p_i$ we need to ensure the observation $\neg p_j$. 
Technically, when $\iseff(p_i,a)$ takes the value \textit{true}, then the observation $\obs(\neg p_j,\tim(p_i,a)+\epsilon)$
needs to be dynamically added. The time of the observation cannot be just $\tim(p_i,a)$, as we first need to assert $p_i$ and one $\epsilon$ later observe $\neg p_j$. Adding the variables and constraints for this new observation during the CSP search is trivial for {\em Dynamic CSPs} (DCSPs)~\cite{mittal1990dynamic}. Otherwise, we need to statically define a new type of observation $\obs(p_i,a,\neg p_j)$, where $a$ supports $p_i$ which is mutex with $p_j$ and, consequently, we will need to observe $\neg p_j$. The difference \textit{w.r.t.} an original $\obs$ is twofold: i) the observation time is now initially unknown, and ii) the observation will be activated or not according to the following constraint:
\newline

{\scriptsize 
	\textbf{if} ($\iseff(p_i,a)$=\textit{true}) \textbf{then} ($\start(\obs(p_i,a,\neg p_j))=\tim(p_i,a)+\epsilon$) \textbf{AND}
	
	\hspace{2.85cm}($\iscond(\neg p_j,\obs(p_i,a,\neg p_j))$=\textit{true})
	
	\textbf{else} $\iscond(\neg p_j,\obs(p_i,a,\neg p_j))$=\textit{false}
}
\newline

Reasoning on mutex depends on optional input knowledge in $C$ and increases notably the size of our formulation, specially in non-DCSPs, but it is automated together with the creation of all the constraints of Table~\ref{table:constraints}.


\subsection{The heuristics}
\label{sec:implementation}

In a pure satisfaction problem, all possible solutions are equally valid. Although a metric to be optimized allows the user to specify preferences over the space of possible solutions towards the best learning, we have not found a metric yet that guarantees this.
%the best learned models.
Therefore, we have focused on simple heuristics that show effective in the tradeoff quality of learning \textit{vs.} performance. 
Due to the nature of solver-independence of our formulation, we propose the following variable+value ordering heuristics, which do not require changes in the implementation of the CSP solver:

%Our formulation is solver-independent, which means we do not use heuristics that . Although this reduces the overall performance, we are interested in using it as a blackbox that can be easily changed with no modification in our formulation. However, our experimentation showed that the 

\begin{enumerate}
	\item X4 ($\iscond$). True first, which learns the most restrictive model of conditions.

	\item X5 ($\iseff$). False first, which learns a model with the min number of causal links, which reduces the number of side effects.

	\item X8 ($\tim$). Lower values first for negative effects, while upper values first for positive effects. This learns delete and positive effects as $\eff_s$ and $\eff_e$, respectively.

	\item X6 ($\reqs$ and $\reqe$). Lower values first for $\reqs$, while upper values first for $\reqe$. This gives priority to $\cond_i$, trying to keep conditions as long as possible in the model learned.
	
	\item X7 ($\supp$). Lower values first to learn supporters that start earlier in the plan trace.

	\item X3 ($\dur$). Lower values first, which learns a model with the shortest actions.

\end{enumerate}


\section{A UNIFIED FORMULATION FOR PLANNING, VALIDATION AND LEARNING}
\label{sec:usingCPValidation}

Our formulation has been primarily designed to solve the task of \textit{model learning}, but it is strongly connected to the tasks of plan {\em synthesis} and plan {\em validation}. This connection lies on the fact that we can leverage the input knowledge on the planning domain over a wide range of (un)known levels. This section provides an integrative view for these three tasks in a temporal planning setting, although this connection also applies to a classical planning model\footnote{In practice, we can transform our temporal planning model into a classical one by setting for every action $\dur(a)=1$ and adding the extra constraint $\start(a)=\en(a)=\reqs(?,a)=\reqe(?,a)=\tim(?,a)$.}, that is, the vanilla model of planning where actions are instantaneous
%~\cite{geffner2013concise}
and a solution is a totally ordered sequence of actions~\cite{geffner2013concise}.


\paragraph{Plan synthesis}

Given a learning task $\mathcal{L}=\tup{F,I,G,A?,O,C}$, each action in $A?$ is completely specified.
This is equivalent to know in advance the values for variables \{X3,X4,X5\} of Table~\ref{table:variables} and the OR-constraint that holds in \{C13,C14\} of Table~\ref{table:21constraints}, while other variables/constraints remain open/unknown. The observations in $O$ are incomplete as no information on the start/end time of actions is given, i.e. the values of \{X1,X2\} are to be determined.
Roughly speaking, this provides the complete model of actions, with the duration, conditions and effects (and their temporal annotation), as defined in the PDDL2.1 domain.
In this case, solving the resulting CSP is equivalent to solve the temporal planning problem $P=\tup{F,I,G,A?}$. The solution is a plan that reaches $G$ from state $I$ under the complete action model defined in $A?$. 
Also, the observations over a plan trace in $O$ can be understood as a sequence of time-stamped {\em landmarks}~\cite{hoffmann2004ordered} for $P$ that are given as input (the predicates of the sets in $O$ must be achieved by any plan that solves $P$ and at the time-stamps given by $O$).

Moreover, it is important to note that our formulation allows us to synthesize a plan despite some of the variables that represent the conditions, effects and duration of an action are unknown. This subsumes the capabilities of off-the-shelf planners that require the complete model of actions for planning.


\paragraph{Plan validation}

Given a learning task $\mathcal{L}=\tup{F,I,G,A?,O,C}$, each action in $A?$ is completely specified like in plan synthesis. The observations over a plan trace in $O$ are specified like in plan synthesis and, additionally, the observations on the start/end times of actions are also complete, which means that the values of \{X1,X2\} are now known.
This provides both the complete model of actions, as defined in the planning domain, and the complete plan trace for all actions (the temporal plan $\pi=\{(a_1,t_{a_1}),(a_2,t_{a_2})\ldots (a_n,t_{a_n})\}$ is consequently known). This allows us to know in advance the values of all variables \{X1,X2$\ldots$ X8\}.
In this case, solving the resulting CSP is equivalent to check whether this full assignment of the CSP is consistent, which means validating the plan $\pi$. In the event of {\em inconsistency}, $\pi$ will need to be executed from $I$ until a condition is unsatisfied to identify the source of the plan failure.

Moreover, our formulation allows us to validate plans despite some of the variables that represent the conditions, effects and duration of an action are unknown, and despite some $(a,t_{a})$ pairs in the input plan trace are incomplete. In such scenarios, the plan validation ability of our formulation is beyond the functionality of VAL (the standard plan validation tool~\cite{howey2004val}) since it can address plan validation of partial, or even empty, action models and with partially observed plan traces. On the contrary, VAL requires both a full plan and a full action model for plan validation.


\paragraph{Action model learning}

Given a learning task $\mathcal{L}=\tup{F,I,G,A?,O,C}$, we can learn the action model from scratch by using our formulation, which is the most expensive and less scalable task; it requires finding a solution to the resulting CSP that builds the full action model.
But we can also learn from several partially specified action models, thus simplifying the learning task. Solving the resulting CSP in those cases is equivalent to \textit{fill the gaps} of a partial action model where, optionally:

\begin{itemize}
	\item Some conditions and/or effects are known (some \{X4,X5\} have initial values); e.g. we know that an action requires for sure certain conditions and we are mainly interested in learning their temporal annotations (\textit{at start}, \textit{over all} or \textit{at end}).  
	
	\item The temporal annotation of all (or just a few) conditions and/or effects is known (we know some OR-constraints that hold in \{C13,C14\}); e.g. we know that some effects always happen \textit{at end} of the action. 
	
	\item Some start/end times or durations are known (some \{X1,X2,X3\} have initial values); e.g. the duration of some actions is known.
\end{itemize}

In addition to the three previous options that simplify and improve the performance of the learning task, there are two additional options that can help us to improve the quality of the learned models where: 

\begin{itemize}
	\item Some $candidates(a)$ are filtered in $a$. If the alphabet $\alpha(a)$ is big, the size of $candidates(a)$ can be huge. This not only makes the learning task more expensive but also facilitates a bad learning of predicates. A \textit{static predicate} represents information that is always true and never changes. For instance, as shown in Section~\ref{section:learningTemporalModels}, \{\texttt{(path loc1 loc2), (link loc1 loc2)}\} represent the fact that there is a path/link between \texttt{loc1} and \texttt{loc2}.	
	They can be necessary in the conditions of some actions (e.g. \texttt{drive-truck}), but never in the effects so they should never learned as effects. Therefore, as a preprocessing stage to improve the input knowledge of $A?$, we can filter static predicates from the effects-set of $candidates(a)$ and make it smaller. 
	%We denote this option as $+F$.
	
	\item The final state observation in $G$ contains a full goal state, i.e. the total assignment of the last state variables, rather than a partial assignment. Similarly to the mutex reasoning, having a full state forces to satisfy the causal links of all state variables, which reduces performance but improves the quality of the learning. Intuitively, since $G$ is more informative now, the learning is better.

\end{itemize}	


%These two options are not only valid for model learning, but also for the synthesis and plan validation tasks.

Finally, it is important to note that our learning task learns one model from one plan trace. A more general learning task could learn one model from dozens of plan traces but, though very appealing, this has serious problems of scalability (the number of variables+constraints grows alarmly). 
To learn from many traces, we can apply the learning task to each individual trace and then return the most learned model, i.e. the most repeated one.
The benefit here is twofold: i) the overall learning time is shorter; and ii) the learned model explains the 100\% of, at least, one plan trace. Although this seems too obvious, it is the main limitation of the learning approaches that learn statistically models from many samples.


 

  
\section{EVALUATION}
\label{sec:evaluation}

Our formulation has been implemented in \textsf{Choco} (\texttt{http://www.choco-solver.org}), an open-source Java library for CP that provides an object-oriented API. 
%to state the constraints to be satisfied. 
\textsf{Choco} uses a static model of variables and constraints, i.e. it is not a DCSP.
We have used this implementation to evaluate the quality of the learned models, from both a syntactic and semantic perspective. We have used four IPC PDDL2.1 domains, thus needing their specific constraints of Section~\ref{section:CSPconstraints}, and solved a collection of instances by using five planners (\textit{LPG-Quality}~\cite{gerevini2003planning}, \textit{LPG-Speed}~\cite{gerevini2003planning}, \textit{TP}~\cite{jimenez2015temporal}, \textit{TFD}~\cite{eyerich2009using} and \textit{TFLAP}~\cite{marzal2016temporal}).
%, where the planning time is limited to 100s. 
We have randomly chosen 50 plans (up to 20-30 actions) per domain, which are automatically compiled into 50 learning tasks that configure our experiments dataset.
Then we solved each learning task, in satisfaction mode, and got the first five action models found. 
We also calculate the most learned models.
The solving time was limited to 300s on an Intel i5-6400 @ 2.70GHz with 8GB of RAM. 

There are several elements in a learning task $\mathcal{L}$ that can be considered, as seen in Sections~\ref{section:CSPconstraints} and \ref{sec:usingCPValidation}.
First, we can enable mutex reasoning or not (named \textbf{mutexON} or \textbf{mutexOFF}).
Second, the static effect predicates can be filtered in $candidates$ of all operators as a preprocessing stage (denoted as \textbf{+F}) or not. 
Third, the final state in $G$ means observing Only partial Goals (named \textbf{OG}) or observing the Full goal State (named \textbf{FS}). 
This way, we run six learning scenarios: OG, OG+F and FS+F, each with mutexON and mutexOFF (we discard FS with no Filtering because of the size of $G$).




\subsection{Experimental setup}

Table~\ref{table:summaryExperiments} summarizes our experiments. \#O is the number of Operators and \#PTL the number of Predicates To Learn. \#candidates is the size of $candidates$ for all the operators when no Filtering (not +F) and when such Filtering (+F) is done. 
Finally, the number of \#tasks solved is given for the six learning scenarios used in this section: OG, OG+F, FS+F, and these in both mutexOFF (first line) and mutexON (second line) versions.
For instance, in {\em zenotravel} we need to learn 5 operators and 28 predicates. In absence of filtering, the number of candidates is 105, which is reduced to 71 when +F. This reduction depends on the domain definition, and ranges from zero ({\em parking}) to significant values ({\em floortile}).
As seen in the table, not all the learning tasks were solved in 300s. This depends on the complexity of the domain, the \#PTL and, specially, the \#candidates. Dealing with a scenario task of OG is easier because there are fewer predicates in $G$ (though this is less informative) than in the FS version. Furthermore, learning with mutexOFF is usually easier than learning with mutexON because no deduced observations are necessary, but less informative.

\begin{table}
	\caption{Summary of our experiments. For the \#tasks solved, the first line reports the results for mutexOFF and the second line for mutexON.}
	\begin{center}
		{\scriptsize 
			%\begin{tabular}{lc|c|cc|c}
			\begin{tabular}{lc|c|cc|p{3.5cm}}
				& \#O & \#PTL  & \multicolumn{2}{c}{\#candidates} & \makebox[3.5cm][c]{\#tasks solved} \\ 

				& & & not +F & +F & \begin{tabular}{p{0.7cm}p{0.7cm}p{0.7cm}} \makebox[0.7cm][c]{\textbf{OG}} & \makebox[0.7cm][c]{\textbf{OG+F}} & \makebox[0.7cm][c]{\textbf{FS+F}} \end{tabular} \\
				%& & & not +F & +F & \begin{tabular}{ccc} OG & OG+F & FS+F \end{tabular} \\
				\cline{2-6}

				zenotravel & 5 & 28 & 105 & 71 & \begin{tabular}{p{0.7cm}p{0.7cm}p{0.7cm}} \makebox[0.7cm][c]{50 \textbf{(0.27)}} & \makebox[0.7cm][c]{50 \textbf{(0.11)}} & \makebox[0.7cm][c]{50 \textbf{(0.04)}} \\ \makebox[0.7cm][c]{50 \textbf{(0.26)}} & \makebox[0.7cm][c]{50 \textbf{(0.21)}} & \makebox[0.7cm][c]{50 \textbf{(0.02)}} \end{tabular} \\
				
				\cline{2-6}
				
				driverlog & 6 & 28 & 144 & 96 & \begin{tabular}{p{0.7cm}p{0.7cm}p{0.7cm}} \makebox[0.7cm][c]{49 \textbf{(0.34)}} & \makebox[0.7cm][c]{50 \textbf{(0.24)}} & \makebox[0.7cm][c]{42 \textbf{(0.28)}} \\ \makebox[0.7cm][c]{42 \textbf{(0.39)}} & \makebox[0.7cm][c]{50 \textbf{(0.33)}} & \makebox[0.7cm][c]{46 \textbf{(0.31)}} \end{tabular} \\				
				
				\cline{2-6}
				
				floortile & 7 & 44 & 417 & 217 & \begin{tabular}{p{0.7cm}p{0.7cm}p{0.7cm}} \makebox[0.7cm][c]{50 \textbf{(0.62)}} & \makebox[0.7cm][c]{50 \textbf{(0.48)}} & \makebox[0.7cm][c]{21 \textbf{(0.48)}} \\ \makebox[0.7cm][c]{28 \textbf{(0.89)}} & \makebox[0.7cm][c]{49 \textbf{(0.97)}} & \makebox[0.7cm][c]{21 \textbf{(0.48)}} \end{tabular} \\				
				
				\cline{2-6}
				
				parking & 4 & 32 & 131 & 131 & \begin{tabular}{p{0.7cm}p{0.7cm}p{0.7cm}} \makebox[0.7cm][c]{50 \textbf{(0.67)}} & \makebox[0.7cm][c]{50 \textbf{(0.60)}} & \makebox[0.7cm][c]{50 \textbf{(0.36)}} \\ \makebox[0.7cm][c]{50 \textbf{(0.82)}} & \makebox[0.7cm][c]{50 \textbf{(0.51)}} & \makebox[0.7cm][c]{50 \textbf{(0.45)}} \end{tabular} \\				
					
				\hline
			\end{tabular}
		}
	\label{table:summaryExperiments}
	\end{center}
\end{table}


In {\em zenotravel} the 50 tasks were solved for all six scenarios. Since five action models are returned per solved task, up to 250 potential different models are to be learned per scenario in {\em zenotravel}. However, this is not the case and this number is usually lower, which is a good indication that models \textit{tend to converge} more easily; i.e. many different tasks learn the same model.
This indication of convergence is depicted in the table between brackets and in bold text, as the relation between the number of different learned models and the potential number of models (clearly lower values are better). In {\em zenotravel} OG+mutexOFF only 67 different models were found, which gives a value of 67/250=0.27. These values are good in {\em zenotravel} (particularly in OG+F and FS+F) and \textit{driverlog}, and worse in \textit{parking} and \textit{floortile}, where OG+F+mutexON shows the worst result (0.97).





\subsection{Syntactic evaluation. Precision and recall}

From a pure syntactic perspective, learning can be considered as an automated design task to create a new model that is similar to a reference (or {\em ground truth}) model. Hence, the aim is to assess the precision and recall of the learned model, two common metrics in learning~\cite{aineto2018icaps,Zhuo2014,ZhuoYHL10}, that give us an intuitive idea on the soundness and completeness, respectively, of the new model.


Given two models, $precision=\frac{p^{=}}{p^{=} + p^{\neg}}$, where $p^{=}$ counts the number of predicates (i.e. conditions+effects) that appear correctly and are temporally annotated equally in both models, and $p^{\neg}$ counts the number of predicates that appear in the learned model but should not appear. On the other hand, $recall=\frac{p^{=}}{p^{=} + p^{\neq}}$, where $p^{\neq}$ counts the number of predicates that should appear in the learned model but are not present. Table~\ref{table:SyntacticResults} depicts these metrics for our six learning scenarios as average scores for all the learned models.
We show the scores for the Start, Invariant and End Conditions (SC, IC and EC respectively), Start and End Effects (SE and EE respectively), and All Conditions and All Effects (AC and AE respectively) for the six learning scenarios (mutexOFF and mutexON are shown in the first and second line, respectively).


The use of mutexON has a positive impact in the precision of AC, but not in AE, and improves the scores of the recall in both AC and AE. Note the recall of SE, which is 0 for mutexOFF in OG and OG+F, and significantly higher when mutexON. With mutexOFF there is no need to learn negative effects, typically modeled as start effects, and the learned models are \textit{relaxed} models where negative effects are not included.
FS generally improves the precision of AC and AE, and also improves the recall of AE because the negative effects present in the full final state cannot be relaxed so need to be learned. Consequently, mutexON or FS help to improve the completeness of the learned effects.
The Filtering scenarios (+F) improves the precision of AC and AE, specially where there exists irrelevant static information (e.g. \textit{floortile}).
Due to lack of space in the table, we do not show all the precision and recall scores for the most learned model. But we do show its precision and recall of AC and AE (between brackets and in bold text). Although the most learned model generally produces scores above the average, this cannot be guaranteed.
%, as a wrong condition or effect could be learned.
Actually, we have detected that sometimes there are \textit{several most learned models}, i.e. different models that are learned the same number of times. But we have not found a safe tie-breaking mechanism to decide 
the model that leads to the best scores. 
%the best model.




%La siguiente tabla es la media de todos los modelos encontrados. 
%\begin{table*}
%\begin{center}
%{\footnotesize 
%\begin{tabular}{lccccc|cc||ccccc|cc}
%	%\hline
%	\multicolumn{15}{l}{a) Only goals} \\
%	& \multicolumn{7}{c}{Precision (mutex OFF \& mutex ON)} & \multicolumn{7}{c}{Recall} (mutex OFF \& mutex ON)  \\	
%	\cline{2-15}
%	& SC & IC & EC & SE & EE & AC & AE & SC & IC & EC & SE & EE & AC & AE \\ 
%	\cline{2-15}
%	
%	\multirow{2}{*}{zenotravel} & 0.62 &	0.23 &	0.94 &	0.80 &	0.81 &	0.60 &	0.81  & 0.09 &	0.92 &	1.0 &	0.0 &	0.72 &	0.64 &	0.36 \\
%	& 0.88 &	0.43 &	0.84 &	0.66 &	0.76 &	0.72 &	0.71 & 0.75	& 0.69 & 1.0 & 0.97 &	0.70 &	0.81 &	0.84 \\
%
%\cline{2-15}
%
%	\multirow{2}{*}{driverlog} & 0.25 &	0.17 &	0.86 &	0.59 &	0.52 &	0.43 & 0.56 &  0.04	& 0.95 &	1.0 &	0.0 &	0.93 & 0.65 &	0.47 \\
%	& 0.60 &	0.20 &	0.74 &	0.66 &	0.28 &	0.51 &	0.47 & 0.71	& 0.60 &	1.0 &	0.92 &	0.7 &	0.77 &	0.81 \\
%\cline{2-15}	
%
%	\multirow{2}{*}{floortile} & 0.17	& 0.15 & 0.44 &	0.44 &	0.25 &	0.25 &	0.35 & 0.02	& 0.96 & 1.0 & 0.0 &	0.68 &	0.65  & 0.34 \\
%	& 0.73	& 0.27 & 0.36 &	0.61 &	0.27 &	0.45 &	0.44 & 0.76 & 0.90 & 1.0 & 0.8 &	0.77 &	0.89 &	0.79 \\
%
%\cline{2-15}
%
%	\multirow{2}{*}{parking} & 0.62 &	0.0	& 0.76 &	0.98 &	0.76 &	0.46 &	0.87 & 0.05 &	1.0 &	1.0	& 0.0 &	0.94 &	0.67 &	0.47 \\
%	& 0.76 &	0.96 &	0.5 &	0.5 &	0.48 &	0.74 &	0.49 & 0.82 &	1.0 &	1.0 &	0.92 &	0.76 &	0.94 &	0.84 \\
%	\hline
%\end{tabular}
%
%\rule{0pt}{8pt}
%
%\begin{tabular}{lccccc|cc||ccccc|cc}
%	%\hline
%	\multicolumn{15}{l}{b) Only goals + filtering of static predicates in effects} \\
%	& \multicolumn{7}{c}{Precision (mutex OFF \& mutex ON)} & \multicolumn{7}{c}{Recall} (mutex OFF \& mutex ON) \\	
%	\cline{2-15}
%	& SC & IC & EC & SE & EE & AC & AE & SC & IC & EC & SE & EE & AC & AE \\ 
%	\cline{2-15}
%	
%	\multirow{2}{*}{zenotravel} & 0.52 &	0.25 &	0.98 &	0.82 &	1.0	& 0.58 &	0.91 & 0.08	& 0.87 & 1.0 &	0.0 &	0.73 &	0.62 &	0.37 \\
%	& 0.82	& 0.38 &	0.74 &	0.65 &	0.89 &	0.65 &	0.77 & 0.67	& 0.47 &	1.0	& 0.95 & 0.68 &	0.71 & 0.82 \\
%	
%
%\cline{2-15}
%
%	\multirow{2}{*}{driverlog} & 0.33 &	0.23 &	0.96 &	0.64 &	0.90 &	0.51 &	0.77 &  0.06 &	0.94 &	1.0 &	0.0 &	0.93 &	0.65 &	0.47 \\
%	& 0.58 &	0.34 &	0.87 &	0.64 &	0.43 &	0.60 &	0.54 &  0.68 &	0.49 &	1.0 &	0.87 &	0.73 &	0.72 &	0.80 \\
%
%\cline{2-15}
%
%	\multirow{2}{*}{floortile} & 0.39 &	0.23 &	0.74 &	0.56 &	0.77 &	0.45 & 0.67 & 0.06	& 0.99 &	1.0	& 0.0	& 0.70	& 0.66 & 0.35 \\
%	& 0.72 &	0.41 &	0.51 &	0.60 &	0.47 &	0.55 & 0.54 & 0.71	& 0.93 &	1.0	& 0.72	& 0.66	& 0.88 & 0.69 \\
%		
%
%\cline{2-15}
%
%	\multirow{2}{*}{parking} & 0.62 &	0.0 &	0.78 &	0.98 &	0.76 &	0.47 &	0.87 &  0.05 &	1.0 &	1.0 &	0.0 &	0.94 &	0.67 &	0.47 \\
%	& 0.76	& 0.99 &	0.48 &	0.5 &	0.48 &	0.74 &	0.49 & 0.8 &	1.0 &	1.0	& 0.9 &	0.75 &	0.93 &	0.83 \\
%	\hline
%\end{tabular}
%
%\rule{0pt}{8pt}
%
%\begin{tabular}{lccccc|cc||ccccc|cc}
%	\multicolumn{15}{l}{c) Full goal state + filtering of static predicates in effects} \\
%	& \multicolumn{7}{c}{Precision (mutex OFF \& mutex ON)} & \multicolumn{7}{c}{Recall} (mutex OFF \& mutex ON) \\	
%	\cline{2-15}
%	& SC & IC & EC & SE & EE & AC & AE & SC & IC & EC & SE & EE & AC & AE \\ 
%	\cline{2-15}
%	
%	\multirow{2}{*}{zenotravel} & 0.84 &	0.42 &	0.90 &	0.71 &	0.99 &	0.72 &	0.85 & 0.70 &	0.60 &	1.0 &	1.0 &	0.75 &	0.53 &	0.88 \\
%	& 0.85 &	0.51 &	0.88 &	0.65 &	0.99 &	0.75 &	0.82 &  0.71 &	0.59 &	1.0 &	0.99 &	0.74 &	0.77 &	0.87 \\
%
%\cline{2-15}
%
%	\multirow{2}{*}{driverlog} & 0.60 &	0.36 &	0.98 &	0.64 &	0.94 &	0.65 &	0.79 & 0.57 &	0.89 &	1.0	& 0.64 &	0.95 &	0.63 &	0.80 \\
%	& 0.62 &	0.34 &	0.69 &	0.69 &	0.57 &	0.55 &	0.63 & 0.74 &	0.48 &	1.0 &	0.96 &	0.83 &	0.74 &	0.90 \\
%	
%
%\cline{2-15}
%
%	\multirow{2}{*}{floortile} & 0.69 &	0.35 &	0.62 &	0.66 &	0.74 &	0.55 &	0.70 & 0.67	& 0.96 &	1.0 &	0.66 &	0.73 &	0.65 &	0.70 \\
%	& 0.70 &	0.45 &	0.39 &	0.59 &	0.67 &	0.51 &	0.63 & 0.86 &	0.88 &	1.0 &	0.89 &	0.87 &	0.91 &	0.88 \\
%	
%
%\cline{2-15}
%
%	\multirow{2}{*}{parking} & 0.86	& 0.0	& 1.0 &	0.89 &	0.89 &	0.62 &	0.89 & 0.53 &	1.0 &	1.0	& 0.72 &	0.95 &	0.67 &	0.84 \\	
%	& 0.74 &	1.0	& 0.50 &	0.50 &	0.46 &	0.75 &	0.48 & 0.87 &	1.0 &	1.0	& 0.94 &	0.82 &	0.96 &	0.88 \\
%	 
%	\hline
%\end{tabular}
%}
%\end{center}
%\end{table*}


\begin{table*}
	\caption{Precision and recall scores.}
	\begin{center}
		{%\footnotesize 
			\scriptsize 
			\begin{tabular}{lccccc|cc||ccccc|cc}
				%\hline
				\textbf{OG} & \multicolumn{7}{c}{Precision; mutexOFF: first line, mutexON: second line} & \multicolumn{7}{c}{Recall; mutexOFF: first line, mutexON: second line}  \\	
				\cline{2-15}
				& SC & IC & EC & SE & EE & AC & AE & SC & IC & EC & SE & EE & AC & AE \\ 
				\cline{2-15}
				
				\multirow{2}{*}{zenotravel} & 0.62 &	0.23 &	0.94 &	0.80 &	0.81 &	0.60 \textbf{(0.76)} &	0.81 \textbf{(1.0)}  & 0.09 &	0.92 &	1.0 &	0.0 &	0.72 &	0.64 \textbf{(0.72)} &	0.36 \textbf{(0.40)} \\
				& 0.88 &	0.43 &	0.84 &	0.66 &	0.76 &	0.72 \textbf{(0.83)} &	0.71 \textbf{(0.88)}  & 0.75	& 0.69 & 1.0 & 0.97 &	0.70 &	0.81 \textbf{(0.89)} &	0.84 \textbf{(0.90)} \\
				
				\cline{2-15}
				
				\multirow{2}{*}{driverlog} & 0.25 &	0.17 &	0.86 &	0.59 &	0.52 &	0.43 \textbf{(0.73)} & 0.56 \textbf{(0.72)} &  0.04	& 0.95 &	1.0 &	0.0 &	0.93 & 0.65 \textbf{(0.73)} &	0.47 \textbf{(0.72)} \\
				& 0.60 &	0.20 &	0.74 &	0.66 &	0.28 &	0.51  \textbf{(0.61)} &	0.47 \textbf{(0.37)} & 0.71	& 0.60 &	1.0 &	0.92 &	0.70 &	0.77 \textbf{(0.81)} &	0.81 \textbf{(0.65)} \\
				\cline{2-15}	
				
				\multirow{2}{*}{floortile} & 0.17	& 0.15 & 0.44 &	0.44 &	0.25 &	0.25 \textbf{(0.38)} &	0.35 \textbf{(0.66)} & 0.02	& 0.96 & 1.0 & 0.0 &	0.68 &	0.65 \textbf{(0.67)} & 0.34 \textbf{(0.34)} \\
				& 0.73	& 0.27 & 0.36 &	0.61 &	0.27 &	0.45 \textbf{(0.36)} &	0.44 \textbf{(0.62)} & 0.76 & 0.90 & 1.0 & 0.80 &	0.77  &	0.89 \textbf{(0.92)} &	0.79 \textbf{(1.0)} \\
				
				\cline{2-15}
				
				\multirow{2}{*}{parking} & 0.62 &	0.0	& 0.76 &	0.98 &	0.76 &	0.46 \textbf{(0.33)} &	0.87 \textbf{(0.85)} & 0.05 &	1.0 &	1.0	& 0.0 &	0.94 &	0.67 \textbf{(1.0)} &	0.47 \textbf{(1.0)} \\
				& 0.76 &	0.96 &	0.50 &	0.50 &	0.48 &	0.74 \textbf{(0.92)} &	0.49 \textbf{(0.54)} & 0.82 &	1.0 &	1.0 &	0.92 &	0.76 &	0.94 \textbf{(0.67)} &	0.84 \textbf{(0.50)} \\
				\hline
			\end{tabular}
			
			\rule{0pt}{8pt}
			
			\begin{tabular}{lccccc|cc||ccccc|cc}
				%\hline
				\textbf{OG+F} & \multicolumn{7}{c}{Precision; mutexOFF: first line, mutexON: second line} & \multicolumn{7}{c}{Recall; mutexOFF: first line, mutexON: second line} 
				\\
				\cline{2-15}
				& SC & IC & EC & SE & EE & AC & AE & SC & IC & EC & SE & EE & AC & AE \\ 
				\cline{2-15}
				
				\multirow{2}{*}{zenotravel} & 0.52 &	0.25 &	0.98 &	0.82 &	1.0	& 0.58 \textbf{(0.75)} &	0.91 \textbf{(1.0)} & 0.08	& 0.87 & 1.0 &	0.0 &	0.73 &	0.62 \textbf{(0.73)} &	0.37 \textbf{(0.35)} \\
				& 0.82	& 0.38 &	0.74 &	0.65 &	0.89 &	0.65 \textbf{(0.53)} &	0.77 \textbf{(0.88)} & 0.67	& 0.47 &	1.0	& 0.95 & 0.68 &	0.71 \textbf{(0.50)} & 0.82 \textbf{(0.80)} \\
				
				
				\cline{2-15}
				
				\multirow{2}{*}{driverlog} & 0.33 &	0.23 &	0.96 &	0.64 &	0.90 &	0.51 \textbf{(0.74)} &	0.77 \textbf{(0.61)} &  0.06 &	0.94 &	1.0 &	0.0 &	0.93 &	0.65 \textbf{(0.67)} &	0.47 \textbf{(0.50)} \\
				& 0.58 &	0.34 &	0.87 &	0.64 &	0.43 &	0.60 \textbf{(0.52)} &	0.54 \textbf{(0.54)} &  0.68 &	0.49 &	1.0 &	0.87 &	0.73 &	0.72 \textbf{(0.56)} &	0.80 \textbf{(0.84)} \\
				
				\cline{2-15}
				
				\multirow{2}{*}{floortile} & 0.39 &	0.23 &	0.74 &	0.56 &	0.77 &	0.45 \textbf{(0.75)} & 0.67 \textbf{(0.80)} & 0.06	& 0.99 &	1.0	& 0.0	& 0.70	& 0.66 \textbf{(0.67)} & 0.35 \textbf{(0.38)} \\
				& 0.72 &	0.41 &	0.51 &	0.60 &	0.47 &	0.55 \textbf{(0.79)} & 0.54 \textbf{(0.42)} & 0.71	& 0.93 &	1.0	& 0.72	& 0.66	& 0.88 \textbf{(0.79)} & 0.69 \textbf{(0.36)} \\
				
				
				\cline{2-15}
				
				\multirow{2}{*}{parking} & 0.62 &	0.0 &	0.78 &	0.98 &	0.76 &	0.47 \textbf{(0.67)} &	0.87 \textbf{(0.85)} &  0.05 &	1.0 &	1.0 &	0.0 &	0.94 &	0.67 \textbf{(0.67)} &	0.47 \textbf{(0.50)} \\
				& 0.76	& 0.99 &	0.48 &	0.50 &	0.48 &	0.74 \textbf{(0.92)} &	0.49 \textbf{(0.47)} & 0.80 &	1.0 &	1.0	& 0.90 &	0.75 &	0.93 \textbf{(1.0)} &	0.83 \textbf{(1.0)} \\
				\hline
			\end{tabular}
			
			\rule{0pt}{8pt}
			
			\begin{tabular}{lccccc|cc||ccccc|cc}
				\textbf{FS+F} & \multicolumn{7}{c}{Precision; mutexOFF: first line, mutexON: second line} & \multicolumn{7}{c}{Recall; mutexOFF: first line, mutexON: second line} 
				\\
				\cline{2-15}
				& SC & IC & EC & SE & EE & AC & AE & SC & IC & EC & SE & EE & AC & AE \\ 
				\cline{2-15}
				
				\multirow{2}{*}{zenotravel} & 0.84 &	0.42 &	0.90 &	0.71 &	0.99 &	0.72 \textbf{(0.83)} &	0.85 \textbf{(0.88)} & 0.70 &	0.60 &	1.0 &	1.0 &	0.75 &	0.53 \textbf{(0.89)} &	0.88 \textbf{(0.90)} \\
				& 0.85 &	0.51 &	0.88 &	0.65 &	0.99 &	0.75 \textbf{(0.83)} &	0.82 \textbf{(0.88)} &  0.71 &	0.59 &	1.0 &	0.99 &	0.74 &	0.77 \textbf{(0.89)} &	0.87 \textbf{(0.90)} \\
				
				\cline{2-15}
				
				\multirow{2}{*}{driverlog} & 0.60 &	0.36 &	0.98 &	0.64 &	0.94 &	0.65 \textbf{(0.67)} &	0.79 \textbf{(0.84)} & 0.57 &	0.89 &	1.0	& 0.64 &	0.95 &	0.63 \textbf{(0.83)} &	0.80 \textbf{(0.84)} \\
				& 0.62 &	0.34 &	0.69 &	0.69 &	0.57 &	0.55 \textbf{(0.44)} &	0.63 \textbf{(0.61)} & 0.74 &	0.48 &	1.0 &	0.96 &	0.83 &	0.74 \textbf{(0.86)} &	0.90 \textbf{(0.88)} \\
				
				
				\cline{2-15}
				
				\multirow{2}{*}{floortile} & 0.69 &	0.35 &	0.62 &	0.66 &	0.74 &	0.55 \textbf{(0.37)} &	0.70 \textbf{(0.84)} & 0.67	& 0.96 &	1.0 &	0.66 &	0.73 &	0.65 \textbf{(0.93)} &	0.70 \textbf{(1.0)} \\
				& 0.70 &	0.45 &	0.39 &	0.59 &	0.67 &	0.51 \textbf{(0.38)} &	0.63 \textbf{(0.65)} & 0.86 &	0.88 &	1.0 &	0.89 &	0.87 &	0.91 \textbf{(0.92)} &	0.88 \textbf{(1.0)} \\
				
				
				\cline{2-15}
				
				\multirow{2}{*}{parking} & 0.86	& 0.0	& 1.0 &	0.89 &	0.89 &	0.62 \textbf{(0.67)} &	0.89 \textbf{(1.0)} & 0.53 &	1.0 &	1.0	& 0.72 &	0.95 &	0.67 \textbf{(0.87)} &	0.84 \textbf{(0.92)} \\	
				& 0.74 &	1.0	& 0.50 &	0.50 &	0.46 &	0.75 \textbf{(0.58)} &	0.48 \textbf{(0.46)} & 0.87 &	1.0 &	1.0	& 0.94 &	0.82 &	0.96 \textbf{(0.93)} &	0.88 \textbf{(0.85)} \\
				
				\hline
			\end{tabular}
		}
	\label{table:SyntacticResults}
	\end{center}
\end{table*}



\subsection{Semantic evaluation. Validation}

There is not a unique reference model when learning temporal models; e.g. \textit{at start} and \textit{over all} can be interchangeable in some domains, but they are syntactically different. Consequently, a pure syntax-based measure might return misleading results. From this standpoint, the quality of the learned model can be assessed by analyzing the success ratio of the learned model \emph{against} unseen samples of a test dataset, analogously to a classification task. 

Formally, $success\ ratio=\frac{samples^{+}}{|dataset|}$, where $samples^{+}$ counts the number of samples the learned model explains on a test $dataset$. A ratio of 1.0 implies learning a model that explains the full dataset: a solution is found which is consistent with the constraints of the learned model together with the test samples ones. 


Table~\ref{table:SemanticResults} shows the average success ratios, where each learned model is validated against the remaining tasks of the same domain.
For instance, in \textit{zenotravel} there are 50 tasks solved per learning scenario (see Table~\ref{table:summaryExperiments}). The five models of each task should be validated against the 49 remaining tasks, which means a huge evaluation (250*49=12250 instances). For simplicity, we only consider the first model learned per task, and the experiment contains 50*49=2450 evaluations per scenario. 
As usual, mutexOFF and mutexON are shown in the first and second line, respectively
The best results are in \textit{zenotravel} and the worst results in \textit{floortile}, which corresponds with the \textit{convergence tendencies} shown in Table~\ref{table:summaryExperiments}. The semantic evaluation for the most learned model against the remaining tasks is shown between brackets and in bold text. 
Such a model generally produces ratios above the average in most scenarios, but this cannot be guaranteed for all the domains.


\begin{table}
	\caption{Success ratio of the models learned \textit{against} the test dataset.}
	\begin{center}
		{\scriptsize 
			\begin{tabular}{lccccc}
				& \textbf{OG} & \textbf{OG+F} & \textbf{FS+F} \\
				\cline{2-4}
		\multirow{2}{*}{zenotravel} & 0.77 \textbf{(1.0)}	& 0.88 \textbf{(0.80)} & 0.95  \textbf{(1.0)}\\
		& 0.77 \textbf{(1.0)} & 0.91 \textbf{(1.0)} & 1.0 \textbf{(1.0)} \\
		
		\cline{2-4}
	
		\multirow{2}{*}{driverlog} & 0.40 \textbf{(0.63)} & 0.69 \textbf{(0.71)} & 0.52 \textbf{(0.51)} \\	
		& 0.44 \textbf{(0.61)} & 0.67 \textbf{(0.63)} & 0.56 \textbf{(0.69)} \\
										
		\cline{2-4}
		
		\multirow{2}{*}{floortile} & 0.22 \textbf{(0.04)} & 0.29 \textbf{(0.67)} & 0.32 \textbf{(0.60)} \\
		& 0.34 \textbf{(0.41)} &  0.31 \textbf{(0.21)} & 0.31 \textbf{(0.35)} \\
		
		\cline{2-4}
		
		\multirow{2}{*}{parking} & 0.40 \textbf{(0.37)} & 0.40 \textbf{(0.37)} & 0.71 \textbf{(0.92)} \\	
		& 0.52 \textbf{(0.59)} & 0.48 \textbf{(1.0)} & 0.65 \textbf{(1.0)} \\
		
			\hline
			\end{tabular}
}
	\label{table:SemanticResults}
\end{center}
\end{table}


All in all, we have found out the semantic evaluation is, however, a bit convoluted. One subtle syntactic difference might not affect the semantic evaluation (e.g. interchangeable conditions). On the contrary, an effect that is not correctly learned involves a subtle penalization in the syntactic evaluation, but it affects negatively the semantic evaluation (that difference might not explain a huge number of samples, as usually happens in \textit{floortile}). Therefore, we have detected that in some domains the success ratio can also return misleading results.
%Esto pasa a menudo en el dominio floortile, donde la accion de paint-up y paint-down a menudo se aprende erroneamente (por ej. una accion pinta dos tiles simultaneamente). Por tanto, la evaluacion semantica es un poco tricky, complicada y a veces enrevesada.



\section{CONCLUSIONS}
\label{sec:conclusions}

There is a growing interest 
%the planning community 
for learning action models in AI planning due to its application to recognition of past behavior for prediction, decision taking, robotics motion capturing, etc. Learning is appealing because these scenarios include a huge number of tasks
%, sometimes difficult to be described formally, 
which require expert knowledge that is impractical in complex domains.

The general contribution of this paper is a solver-independent CP formulation to learn action models in temporal planning, which is more complex than in classical planning because actions can overlap in different ways.
% and conditions+effects are now temporally annotated.
We have formulated all variables and constraints under a flexible schema that accommodates a high level of expressiveness, where all relations of Allen's algebra for temporal reasoning are supported. 
What is more, we also support different levels of specification of the input knowledge. This knowledge, as a partially specified action model, 
can be adapted to address not only the learning task but also the planning and validation tasks.
% under an integrated way.

We have proposed variable+value ordering heuristics that prove effective in our experiments, which test different learning scenarios.
As a summary, the results show that reasoning on mutex (mutexON) is more expensive than mutexOFF, but it improves the quality of the learned models. Filtering static predicates reduces the number of decisions to take, simplifies the task and improves the learning.
Observing only partial goal states (OG) is easier than observing full goal states (FS), but the latter provides a more complete learning. 
In general, mutexON or FS help to learn negative effects, which in other scenarios are relaxed and not learned.
FS is useful when (negative) effects do not change frequently throughout the plan trace, whereas mutexON is very useful to learn strong interactions between contradictory predicates (when one is true the other should be immediately and automatically observed as false).

Our formulation can be solved by Satisfiability Modulo Theories, which is part of our current work. As for future work, we want to extend our formulation to learn from intermediate observations (we need to investigate how many and how frequent they must be) and to learn meta-models (as combinations of several learned models).




%\ack We would like to 


\bibliographystyle{ecai}
\bibliography{ecai}
\end{document}
%%%%%%%%%%%%%%%%%%%%%%%%%%%%%%%%%%%%%%%%%%%%%%%%%%%%%%%%%%%%%%%%%%%%%%
