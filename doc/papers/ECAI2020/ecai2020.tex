\documentclass{ecai}

\usepackage{times}
\usepackage{graphicx}
\usepackage{latexsym}

\usepackage{amssymb}
\usepackage{array}

\newcommand{\tup}[1]{{\langle #1 \rangle}}
\newcommand{\pre}{\mathsf{pre}}    % precondition
\newcommand{\eff}{\mathsf{eff}}    % effect
\newcommand{\cond}{\mathsf{cond}}  % condition
\newcommand{\dur}{\mathsf{dur}}    % duration
\newcommand{\obs}{\mathsf{obs}}    % observation
\newcommand{\start}{\mathsf{start}}% start
\newcommand{\en}{\mathsf{end}}     % end
\newcommand{\til}{\mathsf{til}}    % TIL
\newcommand{\supp}{\mathsf{sup}}   % sup
\newcommand{\tim}{\mathsf{time}}   % time
\newcommand{\reqs}{\mathsf{req\_{start}}} % req_start
\newcommand{\reqe}{\mathsf{req\_{end}}}   % req_end
\newcommand{\ini}{\mathsf{init}}   % init
\newcommand{\goal}{\mathsf{goal}}  % goal



\begin{document}
\title{One-Shot Learning of Concurrent Action Models}
\author{Antonio Garrido \and Sergio Jim\'enez}
 
\maketitle

\begin{abstract}
  We present a {\em constraint programming} (CP) formulation for learning models of planning actions that can be executed in parallel and overlap. With the aim of understanding better the relations between the learning of action models, the synthesis of plans and plan validation, the paper studies a singular learning scenario where just a single observation of a plan execution is available (i.e. one-shot). Our CP formulation, inspired by previous approaches for {\em temporal planning},  models time-stamps for actions, {\em causal link} relationships, {\em threats} and effect {\em interferences}. In addition, our CP formulation can accommodate a different range of expressiveness, subsuming the PDDL2.1 temporal semantics, and that is solver-independent (i.e. off-the-shelf CSP solvers can be used for resolution).   
\end{abstract}


\section{Introduction}
{\em Temporal planning} is an expressive planning model that relaxes the assumption of instantaneous actions of {\em classical planning}~\cite{geffner2013concise}. Actions in temporal planning are called {\em durative}, because each action has an associated duration and hence, actions conditions/effects may hold/happen at different times~\cite{fox2003pddl2}. This means that actions in the temporal planning model can be executed in parallel and overlap in several ways~\cite{cushing2007temporal} and that valid solutions for temporal planning instances indicate the precise time-stamp when actions start and end~\cite{howey2004val}.

Despite the potential of state-of-the-art planners, its applicability to the real world is still somewhat limited because of the difficulty of specifying correct and complete planning models~\cite{kambhampati2007model}. The more expressive the planning model is, the more evident becomes this {\em knowledge acquisition bottleneck} that jeopardizes the usability of AI planning technology. This has led to a growing interest in the planning community for the learning of action models. Most approaches for learning planning action models are purely inductive and require large datasets of observations, e.g. thousands of plan observations to compute statistically significant models that minimize some error metric over the input observations~\cite{yang2007learning,MouraoZPS12,zhuo2013action,kuvcera2018louga}. When model learning is considered an optimization task, it however does not guarantee that the learned action models may fail to explain a given input observation or that the states induced by the execution of plans generated with a learned model are incorrect.

This paper follows a different approach and analyzes the application of {\em Constraint Programming} (CP) for the {\em one-shot learning} of temporal action models, that is, for the singular scenario where action models are learned from a single observation of the execution of a temporal plan. The contributions of this work are two-fold:
\begin{enumerate}  
\item This is the first approach for learning action models for temporal planning, where plan observations can refer to the execution of paralel and overlapping actions. This feature makes our approach appealing for learning in multi-agent environments~\cite{furelos2018carpool}. Learning classical action models from sequential plans is a well-studied problem that has been previously addressed by a wide range of different approaches~\cite{arora2018review}. Since pioneering learning systems like ARMS~\cite{yang2007learning}, we have seen systems able to learn action models with quantifiers~\cite{AmirC08,ZhuoYHL10}, from noisy actions or states~\cite{MouraoZPS12,zhuo2013action}, from null state information~\cite{cresswell2013}, or from incomplete domain models~\cite{ZhuoK17,ZhuoNK13}. While learning an action model for classical planning means computing the actions' conditions and effects that are consistent with the input observations, learning {\em temporal action models} requires also: i) identifying how conditions and effects are temporally distributed in the action execution, and ii) estimate the action duration. 
\item Our CP formulation relates the {\em learning} of planning action models with the {\em synthesis} and the {\em validation} of plans since it allows that of-the-shelf CSP solvers can be used for any of these tasks. Further, we show that the plan validation capacity of our CP formulation is beyond the functionality of VAL (the standard plan validation tool~\cite{howey2004val}) since it can address {\em plan validation} of partial (or even an empty) action models and with partially observed plan traces (VAL requires a full plan and a full action model for plan validation). 
\end{enumerate}

As a motivating example, let us assume an observation of the executions of actions in a {\em logistics scenario}. Learning the temporal planning model will allow us: i) to better understand the insights of the logistics in terms of what is possible (or not) and why, because the model is consistent with the observed data; ii) to suggest changes that can improve the model originally created by a human, e.g. re-distributing the actions' conditions, provided they still explain the observations; and iii) to automatically elaborate similar models for similar scenarios, such as public transit for commuters, tourists or people in general in metropolitan areas ---\emph{a.k.a.} smart urban mobility.



\section{Background}
This section formalizes the {\em temporal} planning model that we follow in this work, the hypothesis space for the addressed learning tasks and the sampling space for the {\em one-shots}.

\subsection{Temporal Planning}
\label{sec:temporalplanning}
We assume that states are factored into a set $F$ of Boolean variables. A state $s$ is a time-stamped full assignment of values to these variables. 

A {\em temporal planning problem} is a tuple $\tup{F,I,G,A}$ where the {\em initial state} $I$, is a fully observed state (i.e. $|I|=|F|$) time-stamped with $t=0$; $G \subseteq F$ is a conjunction of goal conditions over $F$ that defines the set of goal states and $A$ represents the set of {\em durative actions}. Durative actions have an associated duration and may have conditions/effects at different times~\cite{vidal2006branching,garrido2009constraint}.

In this work we assume that {\em durative actions} are grounded from {\em action schemes} (also known as {\em operators}) to compactly represent temporal planning problems. PDDL2.1 is a popular language for representing {\em temporal planning} problems and it is the input language for the temporal track of the International Planning Competition~\cite{fox2003pddl2,ghallab2004automated}. PDDL2.1 somewhat restricts the expressiveness of the temporal planning model since it defines {\em durative actions} $a\in A$ with the following elements:
\begin{enumerate}
\item $\dur(a)$, a positive value indicating the {\em duration} of the action.
\item $\cond_s(a), \cond_o(a), \cond_e(a) \subseteq F$ representing the action {\em conditions}. Unlike the \emph{pre}conditions of classical actions, action conditions in PDDL2.1 must hold: before $a$ is executed ({\em at start}), during the entire execution of $a$ ({\em over all}) or when $a$ finishes ({\em at end}), respectively. In the simplest case, $\cond_s(a) \cup \cond_o(a) \cup \cond_e(a) = \pre(a).$\footnote{Note that in classical planning, $\pre(a)=\{p,not-p\}$ is contradictory. In temporal planning, $\cond_s(a)=\{p\}$ and $\cond_e(a)=\{not-p\}$ is a possible situation, though unusual}
\item $\eff_s(a)$ and $\eff_e(a)$ represent the action {\em effects}. In PDDL2.1 effects can happen {\em at start} or {\em at end} of $a$, respectively, and can still be positive or negative. Again, in the simplest case $\eff_s(a) \cup \eff_e(a) = \eff(a)$.
\end{enumerate}

The semantics of a PDDL2.1 {\em durative action} $a\in A$ can be defined in terms of two discrete events, $\start(a)$ and $\en(a)=\start(a)+\dur(a)$. This means that if action $a$ starts on state $s$ with time-stamp $\start(a)$, $\cond_s(a)$ must hold in $s$. Ending action $a$ in state $s'$, with time-stamp $\en(a)$, means $\cond_e(a)$ must hold in $s'$. {\em Over all} conditions must hold at any state between $s$ and $s'$ or, in other words, throughout interval $[\start(a)..\en(a)]$. Likewise, {\em at start} and {\em at end} effects are instantaneously applied at states $s$ and $s'$, respectively ---continuous effects are not considered. 

A {\em temporal plan} is a set of pairs $\tup{(a_1,t_1),(a_2,t_2)\ldots (a_n,t_n)}$. Each $(a_i,t_i)$ pair contains a durative action $a_i$ and a {\em time-stamp} $t_i=\start(a_i)$. The {\em execution} of a temporal plan starting from a given initial state $I$ induces a state sequence formed by the union of all states $\{s_{t_i}, s_{t_i+\dur(a_i)}\}$, where there exists a state $s_{0}=I$, and $G\subseteq s_{end}$, being $s_{end}$ the last state induced by the execution of the plan (therefore sequential plans can be expressed as temporal plans but not the opposite). We say that a temporal plan is a {\em solution} to a given temporal planning problem when its execution, starting from the corresponding initial state, eventually reaches a state $s_{end}$ that meets the goal conditions.

\subsection{The hypothesis space space}
\label{sec:action-space}
The target of the one-shot learning task addressed in this paper is a set of FOL schemes of PDDL2.1 {\em durative} actions. Fig.~\ref{fig:exampleactions2} shows an example of two schemes for PDDL2.1 {\em durative actions} taken from the {\em driverlog} domain. The schema \texttt{board-truck} has a fixed duration while the duration of \texttt{drive-truck} depends on the driving time associated to the two given locations.

\begin{figure}
  \begin{tabular}{p{\textwidth}}
\begin{tiny}    
\begin{verbatim}
(:durative-action board-truck
  :parameters (?d - driver ?t - truck ?l - location)
  :duration (= ?duration 2)
  :condition (and (at start (at ?d ?l)) (at start (empty ?t))
                  (over all (at ?t ?l)))
  :effect (and (at start (not (at ?d ?l))) (at start (not (empty ?t)))
               (at end (driving ?d ?t))))


(:durative-action drive-truck
  :parameters (?t - truck ?l1 - location ?l2 - location ?d - driver)
  :duration (= ?duration (driving-time ?l1 ?l2))
  :condition (and (at start (at ?t ?l1)) (at start (link ?l1 ?l2))
                  (over all (driving ?d ?t)))
  :effect (and (at start (not (at ?t ?l1))) 
               (at end (at ?t ?l2))))
\end{verbatim}
\end{tiny}    
\end{tabular}
\caption{\small Two action schemes for PDDL2.1 durative actions.}
\label{fig:exampleactions2}
\end{figure}

We assume that the set $F$ of Boolean state variables is defined by the instantiation of a given set of predicates $\Psi$. We denote as ${\mathcal I}_{\xi,\Psi}$ the {\em vocabulary} (set of symbols) that can appear in the {\em conditions} and {\em effects} of a given {\em durative} action schema $\xi$. This set is formally defined as the FOL interpretations of the set of predicates $\Psi$, over the action parameters $pars(\xi)$.

For a {\em durative} action schema $\xi$, the size of its space of possible action models is then $D\times 2^{5\times|{\mathcal I}_{\xi,\Psi}}|$ where D is the number of different possible durations for any action shaped by the $\xi$ schema. This space is significantly larger than for learning STRIPS actions~\cite{yang2007learning} where this number is just $2^{2\times|{\mathcal I}_{\xi,\Psi}}|$ because negative effects must also be preconditions of the same action and cannot be positive effects of that action.

The conditions and effects of the schema of a {\em durative} action can be coded by 5 bit-vectors, each of length $|{\mathcal I}_{\xi,\Psi}|$. This means that the {\em Hamming distance} can be used straightforward as a similarity metric for {\em durative} schemes. For instance, to compare a learned action model with respect to a given reference model that serves as baseline. 

\subsection{The sampling space}
\label{sec:sampling-space}
In this work learning examples are the noiseless but partial observations of the execution of a temporal plan starting from a given initial state. This means that if the value of a state variable in $F$ is observed then, that is the actual value of that variable but that not the value of all the state variables can be observed at any time. For instance, just a subset of them is observable because there are associated sensors reporting their value.

PDDL2.2 is an extension of PDDL2.1 that includes the notion of {\em Timed Initial Literal}~\cite{hoffmann2005} ($\til(f,t)$), as a way of defining a fact $f\in F$ that becomes true at a certain time $t$, independently of the actions in the plan. Traditionally TILs are useful to define exogenous happenings; for instance, a time window when a warehouse is open in a logistics scenario, $\til(open,8)$ and $\til(\neg open,20)$. In this work we show that TILs are also suitable to specify observations. The only differnce is that as goals, observations represent conditions that must be {\em supported} by the execution of the plan at a particular time.

\begin{figure}
  \begin{tabular}{p{\textwidth}}
\begin{tiny}    
\begin{verbatim}

\end{verbatim}
\end{tiny}    
\end{tabular}
\caption{\small Example of the obsevation of a plan execution.}
\label{fig:exampletils}
\end{figure}



\section{Learning Action Models}
This section formalizes the learning task we address in this paper and presents our CP formulation for addressing it with off-the-shelf CSP solvers.

\subsection{One-shot learning of concurrent action models}
We define the task of the {\em one-shot learning of temporal action models} as a tuple $\tup{F,I,G,A?,O,C}$, where:

\begin{itemize}
\item $\tup{F,I,G,A?}$ is a {\em temporal planning problem} where actions in $A?$ are partially specified (i.e., the exact {\em conditions/effects} and/or the {\em duration} of actions are unknown). In the worst case, we only know the vocabulary of the symbols that can appear in the {\em conditions/effects} of the actions. Available prior knowledge can be used to bound this vocabulary for a given action schema.
\item $O$ is the {\em observation} of a plan execution. Al least it contains a full observation of the initial state (time-stamped with $t=0$) and a time stamped final observation that equals the goals of the {\em temporal planning problem} $\tup{F,I,G,A?}$.  Additionally, it can also contain time-stamped observations of the traversed intermediate states and about the time when actions started/ended their execution.
\item $C$ is a set of {\em state-constraints} that reflects domain-specific expert knowledge. These constraints allow us to complete the input observation and to prune inconsistent action models. Figure~\ref{fig:example-statecs} show an example of a set of state-constraints for the {\em driverlog} domain.
\end{itemize}

\begin{figure}
  \begin{scriptsize}
  \begin{tabular}{l}
$\forall x_1,y_1,y_2\ \neg at(x_1,y_1)\vee\neg at(x_1,y_2), \neq (y_1,y_2).$\\
$\forall x_1,y_1,y_2\ \neg in(x_1,y_1)\vee\neg in(x_1,y_2), \neq (y_1,y_2).$\\
$\forall x_1,y_1,y_2\ \neg driving(x_1,y_1)\vee\neg driving(x_1,y_2), \neq (y_1,y_2).$\\
$\forall x_1,y_1,y_2\ \neg driving(y_1,x_1)\vee\neg driving(y_2,x_1), \neq (y_1,y_2).$\\    
$\forall x_1,y_1,y_2\ \neg at(x_1,y_1)\vee\neg driving(x_1,y_2).$\\
$\forall x_1,y_1,y_2\ \neg in(x_1,y_1)\vee\neg driving(x_1,y_2).$\\
$\forall x_1,y_1,y_2\ \neg at(x_1,y_1)\vee\neg in(x_1,y_2).$\\
$\forall x_1,y_1\ \neg empty(x_1)\vee\neg driving(y_1,x_1).$    
  \end{tabular}
\end{scriptsize}      
\caption{\small Examples of state-constraints for the {\em driverlog} domain.}
\label{fig:example-statecs}
\end{figure}

A {\em solution} to the {\em one-shot learning of temporal action models} is a fully specified model of temporal actions $\mathcal{A}$ such that: (1), the conditions, effects and duration of the actions is completely specified and (2), it is {\em consistent} with the given inputs $\tup{F,I,G,A?,O,C}$. In other words, for each action $a \in A?$, we have its equivalent version in $\mathcal{A}$ where we have learned $\dur(a)$, $\cond_s(a)$, $\cond_o(a)$, $\cond_e(a)$, $\eff_s(a)$ and $\eff_e(a)$ and such that we can build on top of the actions in $\mathcal{A}$ a valid plan whose execution starts in $i$, can produce the observations in $O$, and that reaches a final state that satisfies $G$. 

\subsection{Constraint satisfaction for learning action models}
\label{subsec:CPformulation}

Given a {\em one-shot learning of temporal action models} task, our approach is to create a CSP whose solution induces an action model that solves the given learning task. Our CP formulation is inspired by previous work on {\em temporal planning as CP}~\cite{vidal2006branching,garrido2009constraint} and it is solver-independent. This means that any off-the-shelf CSP solver that supports the expressiveness of our CP formulation can be used.

\subsubsection{The CSP Variables}
For each action $a$ in $A?$, we create the seven kinds of variables for the CSP specified in Table~\ref{table:variables}. For simplicity, we model time in $\mathbb{Z}^+$ and bound all maximum times to the makespan of the observed plan execution. If the observation of the plan makespan it is not available we consider a long enough domain for durations.

\begin{table}
\setlength\extrarowheight{2pt}  
\begin{scriptsize}
\begin{tabular}{lll}
{\bf Variable} & {\bf Domain} & {\bf Description} \\\hline
$\start(a)$ & \emph{Known/derived value} & Start time for $a$.\\
$\en(a)$ & \emph{Known/derived value} & End time for $a$.\\
$\dur(a)$ & $[1..max(a)]$ & Duration of $a$ where \\
&&$max(a)=makespan-\start(a)$.\\

$\reqs(p,a)$, & $[0..makespan]$ & Interval $[\reqs(p,a)..\reqe(p,a)]$ \\ 
$\reqe(p,a)$ && during which action $a$ requires $p$.\\

$\tim(p,a)$ & $[0..makespan]$ & Time when effect the $p$ of $a$ happens.\\

$\supp(p,a)$ & The action space & There is a causal link $\tup{b_i,p,a}$. 
\end{tabular}
\end{scriptsize}
\caption{\small The CSP variables and their domains.}
\label{table:variables}
\end{table}

The value of the first three kinds of CSP variables $\start(a)$, $\dur(a)$ and $\en(a)$ is either given by the observation $O$ or derived from the expression $\en(a)=\start(a)+\dur(a)$. The remaining CSP variables model the interval when conditions must hold, the time when the effects happen and the causal links. Besides the actions of the given planning problem, we create two {\em dummy} actions:
\begin{itemize}
\item $\ini$, represents the initial state $I$ ($\start(\ini)=0$ and $\dur(\ini)=0$) and has no variables $\supp, \reqs$ and $\reqe$ because it has no conditions. $\ini$ has as many $\tim(p_i,\ini)=0$ as $p_i$ in $I$.
\item $\goal$, represents $G$ ($\start(\goal)=makespan$ and $\dur(\goal)=0$) and $\goal$ has as many $\supp(p_i,\goal)$ and $\reqs(p_i,\goal)=\reqe(p_i,\goal)=makespan$ as $p_i$ in $G$. $\goal$ has no variables $\tim$ as it has no effects.
\end{itemize}  

This formulation allows also to model TILs like any other regular actions. A $\til(f,t)$ can be seen as an additional {\em dummy} action ($\start(\til(f,t))=t$ and $\dur(\til(f,t))=0$) with no conditions and the single effect $f$ that happens at time $t$ ($\tim(f,\til(f,t))=t$). A $\til$ is analogous to $\ini$, as they both represent information that is given at a particular time, but externally to the execution of the plan. 

\subsubsection{The CSP Constraints}
Table~\ref{table:constraints} shows the constraints defined among the CSP variables of Table~\ref{table:variables}. The first three constraints are explicit enough. The fourth constraint models {\em causal links} $\tup{b,p,a}$ (i.e., the time when $b$ supports $p$ must be before $a$ requires $p$). Note that in a causal link $\tup{b,p,a}$, $\tim(p,b) < \reqs(p,a)$ and not $\leq$ because temporal planning assumes an $\epsilon > 0$ as a small tolerance between the time when a given effect $p$ is supported and when it is required~\cite{fox2003pddl2}. Since we model time in $\mathbb{Z}^+$, $\epsilon=1$ and $\leq$ becomes $<$. 

\begin{table}
\setlength\extrarowheight{2pt}    
\begin{scriptsize}
\begin{tabular}{ll}
{\bf Constraint} & {\bf Description} \\\hline

$\en(a)=\start(a)+\dur(a)$ & End time of $a$. \\

$\en(a) \leq \start(\goal)$ & Always $\goal$ is the last action of the plan. \\

$\reqs(p,a) \leq \reqe(p,a)$ & [$\reqs(p,a)..\reqe(p,a)$]\\
& must be a valid interval.\\

if $\supp(p,a)=b$ then & Modeling causal links $\tup{b,p,a}$.\\
$\tim(p,b) < \reqs(p,a)$ & \\

$\forall c \neq a$ that deletes $p$ at time $t$: & Solving threat of $c$ to causal link $\tup{b,p,a}$\\
if $\supp(p,a)=b$ then &  by promotion or demotion.\\
$t < \tim(p,b)$ OR $t > \reqe(p,a)$ & \\

if $a$ requires and deletes $p$: & When $a$ requires and deletes $p$, the effect \\
$\tim(not-p,a) \geq \reqe(p,a)$ & cannot happen before the condition.\\

$\forall a_i,a_j \mid a_i$ supports $p$ and & Solving effect interference ($p$ and $not-p$):\\
$a_j$ deletes $p$: & they cannot happen at the same time.\\
$\tim(p,a_i) \neq \tim(not-p,a_j)$ &
\end{tabular}
\end{scriptsize}
\caption{\small The CSP constraints.}
\label{table:constraints}
\end{table}

The fifth constraint avoids {\em threats} via promotion or demotion~\cite{ghallab2004automated}. The sixth constraint models the fact that when the same action requires and deletes $p$ then the effect cannot happen before the condition. Note the $\geq$ inequality here; this is possible because if one condition and one effect of $a$ happen at the same time, the underlying semantics in planning considers the condition is checked instantly before the effect~\cite{fox2003pddl2}. The seventh constraint deals with the fact that two (possibly equal) actions have contradictory effects. It is important to note that these constraints involve any type of action, including $\ini$ and $\goal$.

In addition to the constraints of Table~\ref{table:constraints} we can add input {\em state-constraint} (if available), like the ones pictured in Figure~\ref{fig:example-statecs}, to our CSP formulation to prune any inconsistent action model.

\subsubsection{The CSP heuristics}
\label{sec:implementation}
Our CSP formulation is solver-independent, which means we do not use heuristics that may require changes in the implementation of the CSP engine. Although this reduces the solver performance, we are interested in using it as a blackbox that can be easily changed with no modification in our formulation. However, the following standard static heuristics for {\em value selection} has shown effective in our experiments:
\begin{enumerate}
\item $\dur(a)$, lower values first, thus applying the principle of the shortest actions that make the learned model consistent.
\item $\reqs(p,a)$ and $\reqe(p,a)$. For $\reqs$, lower values first, whereas for $\reqe$, upper values first. This gives priority to $\cond_o(a)$, trying to keep the conditions as long as possible.  
\item $\tim(p,a)$, for negative effects, lower values first while for positive effects, upper values first. This gives priority to delete effects as $\eff_s(a)$ and positive effects as $\eff_e(a)$. 
\item $\supp(p,a)$, lower values first, thus preferring the supporter that starts earlier in the plan.
\end{enumerate}

\subsection{Specific constraints for the PDDL2.1 model}
\label{sec:PDDL21constraints}
The defined CSP formulation deals with a model for {\em temporal planning} that is more expressive than PDDL2.1. For instance, it allows conditions and effects to happen at any time, even outside the execution of the action. Imagine a condition $p$ that only needs to be maintained for 5 time units before an action $a$ starts (e.g. warming-up a motor before driving): the expression $\reqe(p,a)=\start(a); \reqe(p,a) = \reqs(p,a)+5$ is possible in our CSP formulation. Additionally, we can represent an effect $p$ that happens in the middle of action $a$: $\tim(p,a) = \start(a)+ (\dur(a) / 2)$.

PDDL2.1 restricts the expressiveness of temporal planning in terms of conditions, effects, durations and structure of the actions so by adding extra constraints to our CSP formluatoin we can make it fully PDDL2.1-compliant. Table~\ref{table:21constraints} summarizes these extra constraints. To limit conditions to only be \emph{at start}, \emph{over all} or \emph{at end} of an action execution we add the following constraint, $((\reqs(p,a) = \start(a))$ OR ($\reqs(p,a) = \en(a)))$ AND $((\reqe(p,a) = \start(a))$ OR ($\reqe(p,a) = \en(a)))$. Likewise, to make effects to exclusively happen either\emph{at start} or \emph{at end} of action executions we add the constraints $((\tim(p,a) = \start(a))$ OR $(\tim(p,a) = \en(a)))$. If all effects happen \emph{at start} the duration of the action would be irrelevant and could exceed the plan makespan. To avoid this, for any action $a$, at least one of its effects should happen \emph{at end}: $\sum_{i=1}^{n =|\eff(a)|} \tim(p_i,a) > n \cdot \start(a)$, which guarantees $\eff_e(a)$ is not empty.

\begin{table}
\setlength\extrarowheight{2pt}    
\begin{scriptsize}
\begin{tabular}{ll}
{\bf Constraint} & {\bf Description} \\\hline
$\reqs(p,a) = \start(a)$ OR $\reqs(p,a) = \en(a)$ & Conditions at start.\\
$\reqe(p,a) = \start(a)$ OR $\reqe(p,a) = \en(a)$ & Conditions at end.\\
$\tim(p,a) = \start(a)$ OR $\tim(p,a) = \en(a)$ & Effects at start or at end.\\
$\sum_{i=1}^{n =|\eff(a)|} \tim(p_i,a) > n \cdot \start(a)$ & At least one effect.\\
$\forall a_i,a_j$ instances of the same operator: & Duration of the schema instantiations.\\
$\dur(a_i) = \dur(a_j)$ &\\

$\forall p_i: (\forall a_j: \reqs(p_i,a_j) = \start(a_j))$ OR & Conditions of the schema instantiations.\\
$(\forall a_j: \reqs(p_i,a_j) = \en(a_j))$ & \\

$\forall p_i: (\forall a_j: \reqe(p_i,a_j) = \start(a_j))$ OR  & \\
$(\forall a_j: \reqe(p_i,a_j) = \en(a_j))$ & \\
$\forall p_i: (\forall a_j: \tim(p_i,a_j) = \start(a_j))$ OR  & Effects of the schema instantiations.\\
$(\forall a_j: \tim(p_i,a_j) = \en(a_j))$ &
\end{tabular}
\end{scriptsize}
\caption{\small The CSP constraints for the PDDL2.1.}
\label{table:21constraints}
\end{table}

Note that if a condition is never deleted in a plan, it can be considered an {\em invariant} condition for such a plan that represents {\em static} knowledge; e.g. a link between two locations that makes driving possible, or modeling a petrol station that allows a refuel action in a given location, etc. The constraint to be added for {\em invariant conditions} $p \in \cond_o(a)$ is simply: $((\reqs(p,a) = \start(a))$ AND $(\reqe(p,a) = \en(a)))$, i.e. Surprisingly, invariant conditions are modeled differently depending on the human modeler. See, for instance, \texttt{(link ?from ?to)} of Fig.~\ref{fig:exampleactions2}, which is modeled as an \emph{at start} condition despite: i) the link should be necessary all over the driving, and ii) no action in this domain can be planned to delete that link.
This also happens in the \emph{transport} domain of the IPC, where a refuel action requires to have a petrol station in a location only \emph{at start}, rather than \emph{over all} which makes more sense. This shows that modeling temporal planning tasks is not easy and it highly depends on the human's decision. On the contrary, our formulation checks the invariant conditions and deals with them always in the same coherent way.

Durations in PDDL2.1 can be defined in two different ways. On the one hand, durations can be equal for all grounded actions of the same operator. For instance, any instantiation of \texttt{board-truck} of Fig.~\ref{fig:exampleactions2} will last 2 time units no matter its parameters. To model this we add the constraint: $\forall a_i,a_j$ being instances of the same operator: $\dur(a_i) = \dur(a_j)$. On the other hand, although different instantiations of \texttt{drive-truck} will last different depending on the locations, different occurrences of the same instantiated action will last equal. The constraint to add here is: $\forall a_i,a_j$ being occurrences of the same durative action: $\dur(a_i) = \dur(a_j)$ (this constraint is subsumed by the previous one in the general case where all instances of the same operator have the same duration).

Last but not least, the structure of conditions and effects for all grounded actions of the same operator is constant in PDDL2.1. This means that if \texttt{(empty ?t)} is an \emph{at start} condition of \texttt{board-truck}, it will be \emph{at start} in any of its grounded actions.
Let $\{p_i\}$ be the conditions of an operator and $\{a_j\}$ be the instances of a particular operator. The following constraints are necessary to guarantee a constant structure:

$\forall p_i: (\forall a_j: \reqs(p_i,a_j) = \start(a_j))$ OR $(\forall a_j: \reqs(p_i,a_j) = \en(a_j))$

$\forall p_i: (\forall a_j: \reqe(p_i,a_j) = \start(a_j))$ OR $(\forall a_j: \reqe(p_i,a_j) = \en(a_j))$

And analogously for all effects $\{p_i\}$ and the instances $\{a_j\}$ of an operator:

$\forall p_i: (\forall a_j: \tim(p_i,a_j) = \start(a_j))$ OR $(\forall a_j: \tim(p_i,a_j) = \en(a_j))$



\section{A unified formulation for planning, validation and learning}
\label{sec:usingCPValidation}
The {\em one-shot learning of temporal action models} is related to plan {\em synthesis} and plan {\em valifation}. We explained that adding extra constraints allows us to restrict the temporal expressiveness of the learned model. We show here that we can also restrict the learned model by constraining the variables to known values, which is specially interesting when there is additional information on the temporal model that needs to be represented. For instance, based on past learned models, we may know the precise duration of an action $a$ is 6, or we can figure out that an effect $p$ always happens at end. Our CP formulation can include this by simply adding $\dur(a)=6$ and $\tim(p,a)=\en(a)$, respectively, which is useful to enrich the partially specified actions in $A?$ of the learning task.

In particular, the possibility of adding those constraints is very appealing when used for validating whether a partial action model allows us to learn a consistent model, as we will see in section~\ref{sec:evaluation}. Let us assume that the distribution of all (or just a few) conditions and/or effects is known and, in consequence, represented in the learning task. If a solution is found, then that structure of conditions/effects is consistent for the learned model. On the contrary, if no solution is found that structure is inconsistent and cannot be explained. Analogously, we can represent known values for the durations. If a solution is found, the durations are consistent, and inconsistent otherwise. Hence, we have three options for validating a partial model \emph{w.r.t.}: i) a known structure with the distribution of conditions/effects; ii) a known set of durations; and iii) a known structure plus a known set of durations (i+ii). The first and second option allows for some flexibility in the learning task because some variables remain open. On the contrary, the third option checks whether a learned model can fit the given constraints, thus reproducing a strict plan validation task equivalent to~\cite{howey2004val}.



\section{Evaluation}
\label{sec:evaluation}
The CP formulation has been implemented in \textsf{Choco}\footnote{\texttt{http://www.choco-solver.org}}, an open-source Java library for constraint programming that provides an object-oriented API to state the constraints to be satisfied.

The empirical evaluation of a learning task can be addressed from two perspectives. From a pure syntactic perspective, learning can be considered as an automated design task to create a new model that is similar to a reference (or {\em ground truth}) model. Consequently, the success of learning is an accuracy measure of how similar these two models are, which usually counts the number of differences (in terms of incorrect durations or distribution of conditions/effects). Unfortunately, there is not a unique reference model when learning temporal models at real-world problems. Also, a pure syntax-based measure usually returns misleading and pessimistic results, as it may count as incorrect a different duration or a change in the distribution of conditions/effects that really represent equivalent reformulations of the reference model. For instance, given the example of Fig.~\ref{fig:exampleactions2}, the condition learned \texttt{(over all (link ?from ?to))} would be counted as a difference in action \texttt{drive-truck}, as it is \texttt{at start} in the reference model; but it is, semantically speaking, even more correct. Analogously, some durations may differ from the reference model but they should not be counted as incorrect. As seen in section~\ref{sec:simpleTask}, some learned durations cannot be granted, but the underlying model is still consistent. Therefore, performing a syntactic evaluation in learning is not always a good idea.

From a semantic perspective, learning can be considered as a classification task where we first learn a model from a training dataset, then tune the model on a validation test and, finally, asses the model on a test dataset. Our approach represents a one-shot learning task because we only use one plan sample to learn the model and no validation step is required.
Therefore, the success of the learned model can be assessed by analyzing the success ratio of the learned model \emph{vs.} all the unseen samples of a test dataset. In other words, we are interested in learning a model that fits as many samples of the test dataset as possible. This is the evaluation that we consider most valuable for learning, and define the success ratio as the percentage of samples of the test dataset that are consistent with the learned model. A higher ratio means that the learned model explains, or adequately fits, the observed constraints the test dataset imposes.


\subsection{Learning from partially specified action models}
We have run experiments on nine IPC planning domains. It is important to highlight that these domains are encoded in PDDL2.1, with the number of operators shown in Table~\ref{table:evaluationExperiments}, so we have included the constraints given in section~\ref{sec:PDDL21constraints}. We first get the plans for these domains by using five planners (\textit{LPG-Quality}~\cite{gerevini2003planning}, \textit{LPG-Speed}~\cite{gerevini2003planning}, \textit{TP}~\cite{jimenez2015temporal}, \textit{TFD}~\cite{eyerich2009using} and \textit{TFLAP}~\cite{marzal2016temporal}), where the planning time is limited to 100s.
The actions and observations on each plan are automatically compiled into a CSP learning instance. Then,
%we create the CP formulation and
we run the one-shot learning task to get a temporal action model for each instance, where the learning time is limited to 100s on an Intel i5-6400 @ 2.70GHz with 8GB of RAM.
In order to assess the quality of the learned model, we validate each model \emph{vs.} the other models \emph{w.r.t.} the \emph{struct}ure, the \emph{dur}ation and the \emph{struct}ure+\emph{dur}ation, as discussed in section~\ref{sec:usingCPValidation}.
For instance, the \emph{zenotravel} domain contains 78 instances, which means learning 78 models. Each model is validated by using the 77 remaining models, thus producing 78$\times$77=6006 validations per struct, dur and struct+dur each. The value for each cell is the average success ratio.
In \emph{zenotravel}, the struct value means that the distribution of conditions/effects learned by using only one plan sample is consistent with all the samples used as dataset (100\% of the 6006 validations), which is the perfect result, as also happens in \emph{floortile} and \emph{sokoban} domains.
The dur value means the durations learned explain 68.83\% of the dataset. This value is usually lower because any learned duration that leads to inconsistency in a sample counts as a failure. The struct+dur value means that the learned model explains entirely 35.76\% of the samples. This value is always the lowest because a subtle structure or duration that leads to inconsistency in a sample counts as a failure.
As seen in Table~\ref{table:evaluationExperiments}, the results are specially good, taking into consideration that we use only one sample to learn the temporal action model.
These results depend on the domain size (number of operators, which need to be grounded), the relationships (causal links, threats and interferences) among the actions, and the size and quality of the plans.

\begin{table}
\begin{scriptsize}
\begin{tabular}{l|llrrr}
& {\bf ops} & {\bf ins} & {\bf struct} & {\bf dur} & {\bf struct+dur}  \\\hline

\emph{zenotravel} & 5 & 78 & 100\% & 68.83\% & 35.76\% \\
\emph{driverlog} & 6 & 73 & 97.60\% & 44.86\% & 21.04\% \\
\emph{depots} & 5 & 64 & 55.41\% & 76.22\% & 23.19\% \\
\emph{rovers} & 9 & 84 & 78.84\% & 5.35\% & 0.17\% \\
\emph{satellite} & 5 & 84 & 80.74\% & 57.13\% & 40.53\% \\
\emph{storage} & 5 & 69 & 58.08\% & 70.10\% & 38.36\% \\
\emph{floortile} & 7 & 17 & 100\% & 80.88\% & 48.90\%\\
\emph{parking} & 4 & 49 & 86.69\% & 81.38\% & 54.89\% \\
\emph{sokoban} & 3 & 51 & 100\% & 87.25\% & 79.96\% \\

\end{tabular}
\end{scriptsize}
\caption{\small Number of operators to learn. Instances used for validation. Average success ratio of the one-shot learned model \emph{vs.} the test dataset in different IPC planning domains.}
\label{table:evaluationExperiments}
\end{table}


We have observed that some planners return plans with unnecessary actions, which has a negative impact for learning precise durations.
%Some planners return plans with unnecessary actions, thus making an adequate learning more difficult. In particular, actions that are redundant have a negative impact for learning precise durations.
The worst result is returned in the \emph{rovers} domain, which models a group of planetary rovers to explore the planet they are on. Since there are many parallel actions for taking pictures/samples and navigation of multiple rovers, learning the duration and the structure+duration is particularly complex in this domain.

\subsection{Learning from scratch}

\section{Conclussions}
\label{sec:conclusions}


We have presented a purely declarative CP formulation, which is independent of any CSP solver, to address the learning of temporal action models. Learning in planning is specially interesting to recognize past behavior in order to predict and anticipate actions to improve decisions.
The main contribution is a simple formulation that is automatically derived from the actions and observations on each plan execution, without the necessity of specific hand-coded domain knowledge. It is also flexible to support a very expressive temporal planning model, though it can be easily modified to be PDDL2.1-compliant.
Formal properties are inherited from the formulation itself and the CSP solver. The formulation is correct because the definition of constraints to solve causal links, threats and effect interferences are supported, which avoids contradictions. It is also complete because the solution needs to be consistent with all the imposed constraints, while a complete exploration of the domain of each variable returns all the possible learned models in the form of alternative consistent solutions.


Unlike other approaches that need to learn from datasets with many samples, we perform a one-shot learning. This reduces both the size of the required datasets and the computation time. The one-shot learned models are very good and explain a high number of samples in the datasets used for testing. Moreover, the same CP formulation is valid for learning and for validation, by simply adding constraints to the variables. This is an advantage, as the same formulation allows us to carry out different tasks: from entirely learning, partial learning/validation (structure and/or duration) to entirely plan validation.
According to our experiments, learning the structure of the actions in a one-shot way leads to representative enough models, but learning the precise durations is more difficult, and even impossible, when many actions are executed in parallel.


Finally,
%it is important to note that
our CP formulation can be represented and solved by Satisfiability Modulo Theories, which is part of our current work. As future work, we want to extend our formulation to learn meta-models, as combinations of many learned models, and a more complete action model.
In the latter, rather than using a partially specified set of actions, we want to find out the conditions/effects together with their distribution.
%We aim at finding out the conditions/effects together with their distribution, thus removing constraints from the partially specified set of actions in $A?$.
The underlying idea of finding an action model consistent with all the constraints will remain the same, but the model will need to be extended with additional decision variables and constraints. %$\mathsf{is\_condition(p,a)}$, $\mathsf{is\_effect(p,a)}$ to decide whether $p$ is a condition or effect of action $a$.
This will probably lead to the analysis of new heuristics for resolution.

\bibliographystyle{ecai}
\bibliography{ecai}
\end{document}
%%%%%%%%%%%%%%%%%%%%%%%%%%%%%%%%%%%%%%%%%%%%%%%%%%%%%%%%%%%%%%%%%%%%%%
